\documentclass{article}
\pdfpagewidth=8.5in
\pdfpageheight=11in

\usepackage{WEDTreport}
% Use the postscript times font!
\usepackage{times}
\usepackage{soul}
\usepackage{url}
\usepackage{xcolor}
\usepackage{polski}
\usepackage[polish]{babel}
\usepackage[utf8]{inputenc}
\usepackage[T1]{fontenc}
\usepackage[utf8]{luainputenc}
\usepackage[hidelinks]{hyperref}
\usepackage[utf8]{inputenc}
\usepackage{caption}
\usepackage{indentfirst}
\usepackage{graphicx}
\usepackage{amsmath}
\usepackage{siunitx}
\usepackage{booktabs}
\usepackage{subfig}

\urlstyle{same}
	
\title{Wprowadzenie do eksploracji danych tekstowych w sieci WWW\\ Odpowiadanie na pytania}

\author{
Maria Konieczka, Alicja Poturała, Jakub Sikora
\affiliations
numery albumów: 283410, 283415, 283418 \\
\emails
maria.konieczka.stud@pw.edu.pl, alicja.poturala.stud@pw.edu.pl, jakub.sikora2.stud@pw.edu.pl
}

\newcommand{\todo}[1]{\textcolor{blue}{\textbf{TO DO:} #1}}
\newcommand{\opracowanyartykul}[1]{\textcolor{teal}{\textbf{Artykuł:} #1\\}}

\begin{document}
\maketitle

\section{Wprowadzenie}
\label{sec:wprowadzenie}


Zadanie odpowiadania na pytania (ang. \emph{question answering}) polega na stworzeniu oprogramowania umożliwiającego automatyczne udzielanie odpowiedzi na pytania zadawane przez człowieka w~ języku naturalnym. Zadanie to jest złożeniem problemu analizy języka naturalnego oraz wyszukiwania informacji.

Dyscyplina \emph{question answering} wyodrębniła się z~ informatyki na początku lat 60-tych XX wieku. Wtedy to zaczęły pojawiać się pierwsze systemy odpowiadające na pytania w~ języku angielskim. Początkowo systemy te były dedykowane konkretnym dziedzinom i~ często służyły jako interfejs do operacji na bazach danych. Jednym z~ pierwszych systemów QA były powstały  w~ 1961 system BASEBALL. Potrafił on odpowiadać na pytania związane z~ zeszłorocznymi rozgrywkami amerykańskiej ligi baseballowej. Dziesięć lat później w~1971 powstał system LUNAR, który potrafił odpowiadać na pytania o~ skały pobrane z~ Księżyca podczas misji Apollo11. 

Systemy otwarte, czyli odpowiadające na pytania ogólne, zaczęły być rozwijane około 20 lat później. Pod koniec 1993 roku stworzony został system START, który jako pierwszy korzystał z~ internetu jako źródła wiedzy. Od tego czasu większość znaczących systemów odpowiadania na pytania ogólne korzysta z~ zasobów sieci www. W~ 2006 roku powstała pierwsza wersja superkomputera IBM Watson, który już  w~ 2011 został zwycięzcą teleturnieju \emph{Jeopardy!} wygrywając z~ dotychczas najdłużej niepokonanym człowiekiem.

\subsection{Definicja problemu}\label{subsec:wpr:cel}
Celem projektu jest zaprojektowanie oraz zaimplementowanie własnego systemu odpowiadania na pytania. Postanowiliśmy, że stworzony przez nas system będzie odpowiadał na klika rodzajów pytań ogólnych sformułowanych w~ języku polskim. Uznaliśmy, że próba poradzenia sobie z~ analizą zdań w~ języku fleksyjnym, jakim jest język polski, będzie o~wiele ciekawsza oraz że na rynku cały czas jest niewiele takich systemów. 





\section{Przegląd literatury}
\label{sec:przeglad}
\todo{Opisać swoje artykuły do 8 kwietnia, nie zapominajmy o bibliografi i cytowaniu} 

\opracowanyartykul{answerbus}
AnswerBus~\cite{zheng2002answerbus} jest systemem odpowiadającym na pytania ogólne, bazującym na wyszkuiwaniu informacji, opartym o~analizę zdań. System przyjmuje pytania w~sześciu językach (angielskim, niemieckim, francuskim, hiszpańskim, włoskim i~portugalskim) oraz odpowiada po angielsku. Odpowiedzi wyszukuje z~internetu, posiłkując się pięcioma różnymi wyszukiwarkami internetowymi m.in. Google czy AltaVista.

Przedstawione rozwiązanie składa się z~kilku komponentów, przetwarzających liniowo otrzymane pytanie. W pierwszej kolejności pytanie trafia do modułu rozpoznającego język. Jeżeli pytanie jest po angielsku to trafia ono dalej, natomiast jeżeli nie, to zostaje ono przetłumaczone wykorzystując internetową usługę tłumacza. Pytanie w~języku angielskim jest następnie analizowane, w~celu określenia typu pytania. Na podstawie typu, wybierane są trzy z~pięciu najbardziej odpowiednich przeglądarek. Wybrane rozwiązania są następnie odpytywane o~specjalnie przygotowane zapytanie, a~następnie z~otrzymanych dokumentów tekstowych wybierane są zdania, potencjalnie zawierające odpowiedź na pytanie. Ostatnim krokiem jest ocena wybranych zdań i~zwrócenie najlepszych wyników użytkownikowi~\cite{zheng2002answerbus}.  

System AnswerBus, podobnie jak większość tego typu rozwiązań, określa typ postawionego pytania. Oprócz oczywistych typów, na przykład w~pytaniu \emph{Jak daleko}, określa typ \emph{odległość}, system stara się ocenić dodatkowe parametry typu oczekiwany rząd wielkości (w~odpowiedzi na przykładowe pytanie oczekujemy odległości w~kilometrach a~nie w~centymetrach). Stwierdzenie typu pytania, pozwala na dokładniejszy dobór wyszukiwarki, od bardzo ogólnej jak Google do specjalistycznych jak Yahoo News~\cite{zheng2002answerbus}.

Analiza odpowiedzi sprowadza się do oceny otrzymanego zdania na podstawie treści pytania. Oczekiwana odpowiedź powinna zawierać jak najwięcej wyrazów z~zapytania. Do oceny odpowiedzi, nie są brane pod uwagę wyrazy nie przenoszące szczególnej informacji typu przyimki, zaimki, spójniki czy okoliczniki. Do oceny odpowiedzi stosuje wzór~\ref{eqn:ocena-answerbus},gdzie $q$ to liczba pasujących słów pytania a~$Q$ to liczba wszystkich wyrazów w~odpowiedzi~\cite{zheng2002answerbus}.

\begin{equation}
    \label{eqn:ocena-answerbus}
    q \geq  \sqrt{Q - 1}  + 1
\end{equation}

Oprócz zliczania słów, system wykrywa powiązania pomiędzy rzeczownikami a~zaimkami i~w~momencie wykrycia zaimka, stara się dokonać podstawienia bazując na zdaniu poprzednim. Na ocenę wpływają również obecność synonimów słów kluczowych oraz pozycja wyniku w~wyszukiwarce~\cite{zheng2002answerbus}.

\opracowanyartykul{asmr}
System AskMSR~\cite{brill2002analysis} jest systemem zajmującym się odpowiadaniem na pytania ogólne. W~tym celu przeszukuje on sieć WWW, wykorzystując internetowe wyszukiwarki. Celem autorów w~tym projekcie, było stworzenie systemu odpowiadającego na pytania, bez przeprowadzania zaawansowanej analizy lingwistycznej zarówno pytania,jak i~potenjalnych odpowiedzi. Głównym założeniem był fakt olbrzymiej redundancji odpowiedzi w~sieci WWW.

Zasadniczo, system składa się z~czterech modułów. Pierwszy z~nich na podstawie pytania buduje zapytania do wyszukiwarki internetowej. Każde z~zapytań, zawiera pewną część oryginalnego pytania. Im więcej wyrazów oryginalnego zapytania znajduje się w~przygotowanym zapytaniu, tym z~większą wagą będą brane pod uwagę odpowiedzi na nie otrzymane.Po konstrukcji zapytań, są one wysyłane do wyszukiwarki internetowej. Do dalszych modułów wysyłane są jedynie krótkie wyrywki dokumentu, przesyłane wraz z~linkami~\cite{brill2002analysis}.

Drugim komponent systemu AskMSR zajmuje się wyszukiwaniem \emph{n-gramów} w~zwróconych odpowiedziach. W~tekstach wyszukiwane są tylko unigramy, bigramy oraz trigramy. Każdy z~\emph{n-gramów} jest oceniany na podstawie wcześniejszej oceny zapytania. Do oceny wliczana jest również liczba unikalnych odpowiedzi zwróconych przez system zawierający dany \emph{n-gram}~\cite{brill2002analysis}.

Przygotowane \emph{n-gramy} są przekazywane do trzeciego modułu filtrowania. Moduł filtrowania, na podstawie ręcznie przygotowanych filtrów opartych o~wyrażenia regularne, określa jeden z~siedmiu typów pytania, a~następnie filtruje \emph{n-gramy}, zawężając potencjalny zestaw odpowiedzi. Ostatnim elementem systemu jest moduł łączenia, który wiąże ze sobą nakładające się \emph{n-gramy}, sukcesywnie budując większe zdania, normalnie niemożliwe do zbudowania wyłącznie poprzez tworzenie \emph{n-gramów}. Ocena połączonych \emph{n-gramów} jest określana na podstawie wyżej ocenianego składnika. Łączenie odbywa się do momentu aż nie można łączyć żadynch odpowiedzi~\cite{brill2002analysis}.

Autorzy systemu zwracają uwagę że najwiekszy wpływ na jakość zwracanych odpowiedzi mają wpływ moduły filtrowania oraz łączenia \emph{n-gramów}. System został przetestowany przy pomocy pytań TREC-9~\cite{voorhees2001trec}. Najbardziej problematyczne okazały się pytania rozpoczynające się od \emph{jak?}, natomiast najlepiej radził sobie z~pytaniami rozpoczynającymi się od \emph{kto?}, z~dużą liczbą typowych dla języka słów~\cite{brill2002analysis}.

\section{Projekt rozwiązania}
\label{sec:projekt-rozwiazania}
Korzystając z~wiedzy i~doświadczeń wyniesionych z~sekcji~\ref{sec:przeglad} niniejszego sprawozdania, zdecydowaliśmy się na stworzenie systemu odpowiadającego na pytania ogólne, z~wyszczególnieniem pytań o~fakty dotyczące:
\begin{itemize}
    \item osób,
    \item miejsc,
    \item dat,
    \item cech wielkościowych,
    \item przedmiotów.
\end{itemize}

Bazą więdzy systemu będzie sieć WWW. Do wyszukiwania dokumentów w~sieci~WWW, wykorzystane zostaną wyszukiwarki internetowe: \emph{Google\footnote{www.google.pl}}, \emph{Bing\footnote{www.bing.com}}, \emph{Yahoo\footnote{www.search.yahoo.com}} oraz \emph{DuckDuckGo\footnote{duckduckgo.com}}. Podobnie jak we wcześniej omawianym systemie AskMSR~\cite{brill2002analysis}, do dalszej analizy przekazywane będą jedynie \emph{snippety}, co pozwoli na uproszczenie etapu filtracji akapitów.

Odpowiedzią na przedstawione pytanie, będzie prosta odpowiedź typu \emph{Named-Enitity}. Nie przewidujemy odpowiadania na pytania w~języku naturalnym.

\subsection{Zarys algorytmu}
Na rysunku~\ref{fig:algorithm-overview}, przedstawiony został ogólny schemat algorytmu odpowiadania na pytania.

\begin{figure}[h]
    \centering
    \includegraphics[width=0.7\columnwidth]{figures/WEDT-Algorytm.pdf}
    \caption{Ogólny schemat algorytmu odpowiadania na pytania}
    \label{fig:algorithm-overview}
\end{figure}

Pierwszym krokiem algorytmu jest akwizycja pytania od użytkownika systemu. Każde pytanie będzie analizowane w~sposób indywidualny, bez przechowywania wcześniejszych pytań i~utrzymywania kontekstu. Nie zakładamy również żadnego profilowania użytkowników, ponieważ typowo proste pytania o~fakty, są ze sobą słabo powiązane.

W~kolejnym kroku, strumień sterowania rozdwaja się i~jest przekazywany do dwóch modułów, które przetwarzają zapytanie równolegle. Moduł akwizycji wiedzy, na podstawie przekazanego pytania, stara się uzyskać jak największą ilość \emph{snippetów}, generując specjalistyczne zapytania do każdej z~czterech wyszukiwarek, tak jak to zostało przedstawione w~\cite{zheng2002answerbus}. Oprócz tego, aby wykorzystać redundantość sieci~WWW, do wyszukiwarek wysłane zostaną również okrojone zapytania, pozbawione części słów, tak aby zwiększyć zakres poszukiwań, podobnie jak w~\cite{brill2002analysis}. Aby zwiększyć rozmiar zbioru wyszukanych \emph{snippetów}, słowa kluczowe w~zapytaniu, będą wymieniane na ich synonimy~\cite{przybyla-2013-question}.

W~trakcie wyszukiwania \emph{snippetów}, w~drugiej gałęzi algorytmu, następuję klasyfikacja typu pytania i~oczekiwanej odpowiedzi. Klasy pytań są definiowane przez zaimki pytające, podobnie jak w~\cite{gupta2012survey}. Wyróżniamy pytania typu: \emph{Kto?}, \emph{Jaki/Jaka/Jakie?}, \emph{Jak?}, \emph{Gdzie?}, \emph{Kiedy?}, \emph{Co?} oraz \emph{Który/Która/Które?}. Wykrywanie typu pytania odbywać się będzie w~prosty sposób, stosując reguły podstawieniowe oraz rozkład zdania. Określanie typu oczekiwanej odpowiedzi będzie wynikało głównie z~typu pytania, wspartej możliwością odpytania usługi \emph{plWordNet}~\cite{MazPiaRudSzpaKedz:16} oraz dodatkowym sprawdzeniem wyrażeń regularnych.

Każdy wyszukany \emph{snippet} zostanie poddany analizie morfologicznej, po której przyporządkowana zostanie mu odpowiednia ocena. W~każdym zdaniu, za pomocą taggera oraz \emph{plWordNetu} wyszukiwane będą \emph{Named-Entities}, które mogą stanowić potencjalną odpowiedź na pytanie. Ostatnim krokiem procesu będzie zwrócenie najlepszej odpowiedzi lub zbioru najlepszych odpowiedzi. Do odpowiedzi dołączony może zostać link ze źródłem, w~celu umożliwienia użytkownikowi systemu osobistej weryfikacji odpowiedzi.

\subsection{Dekompozycja systemu}
Zaprojektowany system składa się z~kilku funkcjonalnych komponentów, z~czego tylko część zostanie przez nas zrealizowana w~całości. Duży fragment systemu będzie wykorzystywał gotowe komponenty takie jak \emph{plWordNet} czy wyszukiwarki internetowe. Na rysunku \ref{fig:system-components} przedstawiony został podział na komponenty funkcjonalne, ze szczególnym wyróżnieniem relacji pomiędzy nimi.

\begin{figure}[h]
    \centering
    \includegraphics[width=\columnwidth]{figures/WEDT-Komponenty.pdf}
    \caption{Podział systemu KPSAnswer na funkcjonalne komponenty}
    \label{fig:system-components}
\end{figure}

Planowany podział zakłada istnienie dwóch, głównych komponentów:
\emph{KPSAnswer System\footnote{Nazwa KPS pochodzi od pierwszych liter nazwisk autorów}} oraz wyszukiwarki podsumowań, wspieranych kilkoma komponentami pobocznymi.
Komponent \emph{KPSAnswer System} zajmuje się generacją odpowiedzi na pytania przychodzące z~zewnątrz, korzystając z~interfejsu \texttt{Pytania}. W~celu wygenerowania odpowiedzi, komponent ten odpytuje \texttt{Wyszukiwarkę podsumowań} przy użyciu interfejsu \texttt{Dokumenty} o~zbiór podsumowań, mogących zawierać odpowiedź. \texttt{Wyszukiwarka podsumowań} zdobywa podsumowania odpytując strony implementujące interfejs \texttt{Wyszukiwarka internetowa}.

\subsection{Propozycja implementacji}
Wydzielenie dwóch głównych komponentów, komunikujących się z~komponentami pobocznymi za pomocą interfejsów komunikującyjnych, pozwala na wydzielenie bytów, aplikacji w~systemie. Proponowany przez nas podział na aplikacje, przedstawiony został na rysunku~\ref{fig:system-deployment}.

\begin{figure}[h!]
    \centering
    \includegraphics[width=0.8\columnwidth]{figures/WEDT-Deployment.pdf}
    \caption{Diagram rozmieszczenia systemu KPSAnswer}
    \label{fig:system-deployment}
\end{figure}

Na maszynie PC z~systemem operacyjnym Linux uruchomione zostaną dwa procesy, które przygotwane zostaną przy użyciu języka wysokiego poziomu Python w~wersji~3. Pierwszy z~nich będzie przyjmował pytania za pomocą wystawionego interfejsu REST API. Dzięki temu, aplikacja będzie mogła być odpytywana za pomocą wiersza poleceń, programów typu Postman lub dedykowanego interfejsu graficznego. Drugi proces odpowiedzialny będzie za przeszukiwanie sieci WWW, generację odpowiednich zapytań oraz odpytywanie wyszukiwarek i~agregację podsumowań. Obie aplikacje będą komunikowały się z~aplikacjami w~sieci WWW, dlatego też maszyna będzie musiała być cały czas podłączona do internetu. Zastosowanie komunikacji z~użyciem protokołu HTTP pozwoli na potencjalne uruchomienie systemu w~chmurze, tak jak system Watson, omawiany w~\cite{lapshin2012question}.

\section{Dane testowe}
\label{sec:dane-testowe}
\todo{Opisać charakterystykę daynch testowych, jak ogólnie wyglądają, że są raczej krótkie i na zadany temat}
Dane są jakie są i elo

\section{Podsumowanie}
\label{sec:podsumowanie}
\todo{Krótkie podsumowanie, w sumie nie wiem czy bedziemy to chcieli pisać ale głupio tak skonczyć bez podsumowania}

\bibliographystyle{abbrv}
\bibliography{bibliography}

\end{document}