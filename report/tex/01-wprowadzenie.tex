

Zadanie odpowiadania na pytania (ang. \emph{question answering}) polega na stworzeniu oprogramowania umożliwiającego automatyczne udzielanie odpowiedzi na pytania zadawane przez człowieka w~ języku naturalnym. Zadanie to jest złożeniem problemu analizy języka naturalnego oraz wyszukiwania informacji.

Dyscyplina \emph{question answering} wyodrębniła się z~ informatyki na początku lat 60-tych XX wieku. Wtedy to zaczęły pojawiać się pierwsze systemy odpowiadające na pytania w~ języku angielskim. Początkowo systemy te były dedykowane konkretnym dziedzinom i~ często służyły jako interfejs do operacji na bazach danych. Jednym z~ pierwszych systemów QA były powstały  w~ 1961 system BASEBALL. Potrafił on odpowiadać na pytania związane z~ zeszłorocznymi rozgrywkami amerykańskiej ligi baseballowej. Dziesięć lat później w~1971 powstał system LUNAR, który potrafił odpowiadać na pytania o~ skały pobrane z~ Księżyca podczas misji Apollo11. 

Systemy otwarte, czyli odpowiadające na pytania ogólne, zaczęły być rozwijane około 20 lat później. Pod koniec 1993 roku stworzony został system START, który jako pierwszy korzystał z~ internetu jako źródła wiedzy. Od tego czasu większość znaczących systemów odpowiadania na pytania ogólne korzysta z~ zasobów sieci www. W~ 2006 roku powstała pierwsza wersja superkomputera IBM Watson, który już  w~ 2011 został zwycięzcą teleturnieju \emph{Jeopardy!} wygrywając z~ dotychczas najdłużej niepokonanym człowiekiem.

\subsection{Cel projektu}\label{subsec:wpr:cel}
Celem projektu jest zaprojektowanie oraz zaimplementowanie własnego systemu odpowiadania na pytania. Postanowiliśmy, że stworzony przez nas system będzie odpowiadał na klika rodzajów pytań ogólnych sformułowanych w~ języku polskim. Uznaliśmy, że próba poradzenia sobie z~ analizą zdań w~ języku przypadkowym, jakim jest język polski, będzie o~wiele ciekawsza oraz że na rynku cały czas jest niewiele takich systemów. 



