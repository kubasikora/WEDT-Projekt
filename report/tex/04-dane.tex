W~celu weryfikacji poprawności naszego rozwiązania, przygotowaliśmy zbiór danych testowych, składający się ponad 250 pytań o~ogólnej tematyce. Przygotowując pytania, staraliśmy się równomiernie podzielić zbiór pomiędzy pięc typów potencjalnej odpowiedzi. Pytania pochodziły głównie z~popularnych teleturniejów takich jak Milionerzy czy Jeden z~dziesięciu ale również z~innych popularnych quizów internetowych oraz konkursów wiedzy dla młodzieży.

\begin{figure}[h!]
    \begin{tikzpicture}
        \pie [rotate = 180]
        {20.3/OSOBA,
         19.9/MIEJSCE,
         19.6/DATA,
         20.3/WIELKOŚĆ,
         19.9/RZECZ}
    \end{tikzpicture}
    \label{fig:rozklad-typow-odpowiedzi}  
    \caption{Rozkład typów oczekiwanych odpowiedzi w~przygotowanej bazie pytań}
\end{figure}

Z~powodu charakteru pytań o~fakty, większość oczekiwanych odpowiedzi to rzeczowniki lub liczebniki. Oprócz tego, pojawią się również pytania o~cechy: przymiotniki i~przysłówki.

Duży zbiór danych pozwoli nam na przeprowadzenie badań nad poprawnością odpowiedzi systemu. Oprócz gotowego systemu, powstanie moduł badający procent poprawnych odpowiedzi w~sposób automatyczny. Aby tak badać poprawność, moduł ten musi mieć dostęp do poprawnych odpowiedzi. Co więcej, odpowiedź w~systemie powinna być przechowywana we wszystkich możliwych formach, ponieważ system może zwracać odpowiedź w~dowolnym przypadku, osobie czy liczbie. Problem ten został szerzej opisany w~artykule~\cite{brill2002analysis}.

Węzeł do automatycznego badania poprawności odpowiedzi pozwoli nam na przeprowadzenie dodatkowych badań nad zasadnością pewnych elementów w~systemie oraz wartości parametrów modułów. Modularność projektowanego systemu pozwoli nam na wyłączanie pewnych komponentów takich jak słownik synonimów, dzięki czemu będziemy mogli znaleźć optymalny kompromis pomiędzy dokładnością odpowiedzi a~czasem oczekiwania na odpowiedź.

\begin{figure}[h!]
    \begin{tikzpicture}
        \pie [rotate = 180]
        {62.5/Rzeczowniki,
         29.8/Liczebniki,
         7.7/Przymiotniki/przysłówki}
    \end{tikzpicture}
    \label{fig:rozklad-typow-odpowiedzi}  
    \caption{Rozkład części mowy oczekiwanych odpowiedzi w~przygotowanej bazie pytań}
\end{figure}