\todo{Opisać swoje artykuły do 8 kwietnia, nie zapominajmy o bibliografi i cytowaniu} 

\opracowanyartykul{answerbus}
AnswerBus~\cite{zheng2002answerbus} jest systemem odpowiadającym na pytania ogólne, bazującym na wyszkuiwaniu informacji, opartym o~analizę zdań. System przyjmuje pytania w~sześciu językach (angielskim, niemieckim, francuskim, hiszpańskim, włoskim i~portugalskim) oraz odpowiada po angielsku. Odpowiedzi wyszukuje z~internetu, posiłkując się pięcioma różnymi wyszukiwarkami internetowymi m.in. Google czy AltaVista.

Przedstawione rozwiązanie składa się z~kilku komponentów, przetwarzających liniowo otrzymane pytanie. W pierwszej kolejności pytanie trafia do modułu rozpoznającego język. Jeżeli pytanie jest po angielsku to trafia ono dalej, natomiast jeżeli nie, to zostaje ono przetłumaczone wykorzystując internetową usługę tłumacza. Pytanie w~języku angielskim jest następnie analizowane, w~celu określenia typu pytania. Na podstawie typu, wybierane są trzy z~pięciu najbardziej odpowiednich przeglądarek. Wybrane rozwiązania są następnie odpytywane o~specjalnie przygotowane zapytanie, a~następnie z~otrzymanych dokumentów tekstowych wybierane są zdania, potencjalnie zawierające odpowiedź na pytanie. Ostatnim krokiem jest ocena wybranych zdań i~zwrócenie najlepszych wyników użytkownikowi~\cite{zheng2002answerbus}.  

System AnswerBus, podobnie jak większość tego typu rozwiązań, określa typ postawionego pytania. Oprócz oczywistych typów, na przykład w~pytaniu \emph{Jak daleko}, określa typ \emph{odległość}, system stara się ocenić dodatkowe parametry typu oczekiwany rząd wielkości (w~odpowiedzi na przykładowe pytanie oczekujemy odległości w~kilometrach a~nie w~centymetrach). Stwierdzenie typu pytania, pozwala na dokładniejszy dobór wyszukiwarki, od bardzo ogólnej jak Google do specjalistycznych jak Yahoo News~\cite{zheng2002answerbus}.

Analiza odpowiedzi sprowadza się do oceny otrzymanego zdania na podstawie treści pytania. Oczekiwana odpowiedź powinna zawierać jak najwięcej wyrazów z~zapytania. Do oceny odpowiedzi, nie są brane pod uwagę wyrazy nie przenoszące szczególnej informacji typu przyimki, zaimki, spójniki czy okoliczniki. Do oceny odpowiedzi stosuje wzór~\ref{eqn:ocena-answerbus},gdzie $q$ to liczba pasujących słów pytania a~$Q$ to liczba wszystkich wyrazów w~odpowiedzi~\cite{zheng2002answerbus}.

\begin{equation}
    \label{eqn:ocena-answerbus}
    q \geq  \sqrt{Q - 1}  + 1
\end{equation}

Oprócz zliczania słów, system wykrywa powiązania pomiędzy rzeczownikami a~zaimkami i~w~momencie wykrycia zaimka, stara się dokonać podstawienia bazując na zdaniu poprzednim. Na ocenę wpływają również obecność synonimów słów kluczowych oraz pozycja wyniku w~wyszukiwarce~\cite{zheng2002answerbus}.

\opracowanyartykul{asmr}
System AskMSR~\cite{brill2002analysis} jest systemem zajmującym się odpowiadaniem na pytania ogólne. W~tym celu przeszukuje on sieć WWW, wykorzystując internetowe wyszukiwarki. Celem autorów w~tym projekcie, było stworzenie systemu odpowiadającego na pytania, bez przeprowadzania zaawansowanej analizy lingwistycznej zarówno pytania,jak i~potenjalnych odpowiedzi. Głównym założeniem był fakt olbrzymiej redundancji odpowiedzi w~sieci WWW.

Zasadniczo, system składa się z~czterech modułów. Pierwszy z~nich na podstawie pytania buduje zapytania do wyszukiwarki internetowej. Każde z~zapytań, zawiera pewną część oryginalnego pytania. Im więcej wyrazów oryginalnego zapytania znajduje się w~przygotowanym zapytaniu, tym z~większą wagą będą brane pod uwagę odpowiedzi na nie otrzymane.Po konstrukcji zapytań, są one wysyłane do wyszukiwarki internetowej. Do dalszych modułów wysyłane są jedynie krótkie wyrywki dokumentu, przesyłane wraz z~linkami~\cite{brill2002analysis}.

Drugim komponent systemu AskMSR zajmuje się wyszukiwaniem \emph{n-gramów} w~zwróconych odpowiedziach. W~tekstach wyszukiwane są tylko unigramy, bigramy oraz trigramy. Każdy z~\emph{n-gramów} jest oceniany na podstawie wcześniejszej oceny zapytania. Do oceny wliczana jest również liczba unikalnych odpowiedzi zwróconych przez system zawierający dany \emph{n-gram}~\cite{brill2002analysis}.

Przygotowane \emph{n-gramy} są przekazywane do trzeciego modułu filtrowania. Moduł filtrowania, na podstawie ręcznie przygotowanych filtrów opartych o~wyrażenia regularne, określa jeden z~siedmiu typów pytania, a~następnie filtruje \emph{n-gramy}, zawężając potencjalny zestaw odpowiedzi. Ostatnim elementem systemu jest moduł łączenia, który wiąże ze sobą nakładające się \emph{n-gramy}, sukcesywnie budując większe zdania, normalnie niemożliwe do zbudowania wyłącznie poprzez tworzenie \emph{n-gramów}. Ocena połączonych \emph{n-gramów} jest określana na podstawie wyżej ocenianego składnika. Łączenie odbywa się do momentu aż nie można łączyć żadynch odpowiedzi~\cite{brill2002analysis}.

Autorzy systemu zwracają uwagę że najwiekszy wpływ na jakość zwracanych odpowiedzi mają wpływ moduły filtrowania oraz łączenia \emph{n-gramów}. System został przetestowany przy pomocy pytań TREC-9~\cite{voorhees2001trec}. Najbardziej problematyczne okazały się pytania rozpoczynające się od \emph{jak?}, natomiast najlepiej radził sobie z~pytaniami rozpoczynającymi się od \emph{kto?}, z~dużą liczbą typowych dla języka słów~\cite{brill2002analysis}.

\opracowanyartykul{question answering systems ibm watson}
W pracy~\cite{lapshin2012question} przedstawiona została ogólna struktura systemu do odpowiadania na pytania ogólne. Autor podzielił proces odpowiadania na kilka odrębnych faz:
\begin{itemize}
    \item określanie typu pytania,
    \item przetwarzanie pytania,
    \item określenie kontekstu,
    \item przeszukiwanie bazy wiedzy,
    \item konstruowanie odpowiedzi.
\end{itemize}

Oprócz tego, w~pracy zostały określone obostrzenia, które powinien spełniać system:
\begin{itemize}
    \item interaktywność,
    \item odpowiadanie w~czasie rzeczywistym,
    \item logiczne wnioskowanie,
    \item profilowanie użytkownika,
    \item odpowiadanie w~różnych językach.
\end{itemize}

Przykładowym systemem, spełniającym przedstawione wymagania, jest \emph{DeepQA}, znany obecnie pod nazwą \emph{IBM Watson}. W~pierwszym etapie, system analizuje pytania, określając jego klasę, typ oczekiwanej odpowiedzi oraz buduje związki logiczne pomiędzy wyrazami w~pytaniu. Dzięki temu, jest on w~stanie zbudować odpowiednie zapytanie do bazy wiedzy, którą w~tym przypadku jest sieć WWW. W~kolejnym kroku, \emph{Watson} tworzy hipotezy odpowiedzi, które następnie sprawdza, poszukując dowodów oraz dokonując analizy krytycznej źródeł. Ostatecznie, ze zbioru odpowiedzi wybierana jest ta najbardziej prawdopodobna i~to ona zostaje zwrócona użytkownikowi~\cite{lapshin2012question}.

Najważniejszym elementem całego systemu jest moduł klasyfikacji odpowiedzi. Poprawna klasyfikacja jest odpowiedzialna za największy przyrost poprawnych odpowiedzi. System korzysta ze zbioru kategorii pytań, nazywany ontologią. Pierwsze zbiory określały od 13 do 18 kategorii pytań, natomiast w~późniejszych pracach, zaczęto dzielić kategorię na kategorię główne i~szczegółowe, przykładowo w~\cite{li2002learning} określono sześć typów głównych i prawie pięćdziesiąt typów szczegółowych. Do określania typu pytania najczęściej wykorzystywane są klasyfikatory regułowe oraz statystyczne. Tego typu naiwne podejście jest bardzo proste i~daje dobre rezultaty~\cite{lapshin2012question}.

To co pozwala dobrze oceniać system \emph{IBM Watson} to nie tylko dobre wyniki w~testach na pytaniach z~konferencji TREC, ale również potężny sukces biznesowy, który firma IBM osiągnęła sprzedając swoje rozwiązanie\footnote{https://www.ibm.com/watson}. W~połączeniu z~otwartą chmurą, nowoczesne przedsiębiorstwa mogą wykorzystać Watsona do tworzenia inteligentnych systemów zarządzania oraz wspierania ludzi w~przemyśle, medycynie i~logistyce. Najbardziej znanym przykładem użycia systemu \emph{IBM Watson} jest jego udział w~programie \emph{Jeopardy!}, w~którym to uczestnicy odpowiadają na zróżnicowane i~skomplikowane pytania~\cite{lapshin2012question}. 


\opracowanyartykul{A Practical QA System in Restricted Domains}
Praca ~\cite{restrictedWeather} przedstawia opis systemu odpowiadającego na pytania stworzonego dla robota pracującego w~warunkach domowych. Autorzy zauważyli, że systemy tego typu są wykorzystywane przeważnie w~ograniczonych dziedzinach takich jak pogoda czy program stacji telewizyjnych oraz, że dla użytkowników takich systemów znacznie ważniejsza jest precyzja uzyskanych odpowiedzni. Artykuł skupił swoją uwagę na opisie systemu umożliwiającemy zadawanie pytań na temat pogody. 

Proces odpowiadania na pytania jest trójfazowy: 
\begin{itemize}
	\item przetworzenie zapytania w języku naturalnym na zapytanie SQL,
	\item realizacja zapytania,
	\item konwersja otrzymanego rezultatu na język naturalny.
\end{itemize}
Stworzony system składa się z trzech elementów: 
\begin{itemize}
	\item silnika IE, 
	\item DBMS, 
	\item silnika QA.
\end{itemize}

Dane wykorzystywane do odpowiedzi na pytania pochodzą z~wybranych przez autorów, połowicznie ustrukuryzowanych stron internetowych Korea Meteorological Administration. Moduł silnika IE co godzinę pobiera z~odpowiedniej strony internetowej najnowsze dokumenty, parsuje je w~celu wydobycia zdyskretyzowanych informacji i zapisuje je w bazie danych. 

Silnik QA służy do analizowania zapytań oraz konwertowania uzyskanych odpowiedzi na język naturalny. Autorzy \cite{restrictedWeather} wydzielili skończoną liczbę tematów (związanych z pogodą) o~jakie może zapytać użytkownik. Są to między innymi: temperatura powietrza, prędkość wiatru, opady. Analiza pytania polega na ekstrakcji słów kluczowych dotyczących m.in.: tematu pytania, miejsca, czasu. Aby móc zidentyfikować dokładnie intencje człowieka, przy użyciu specjalnego modułu, wyszczególnione z~zapytania słowa kluczowe są konwertowane na konretne terminy stosowane w systemie. Dzięki temu pytanie o konieczność wzięcia parasola jest traktowane tak samo jak pytanie czy pada aktualnie deszcz. W~kolejnym etapie przetwarzania pytania system zamienia wyrażenia czasowe takie jak \textit{dzisiaj} i~\textit{jutro} na wartości bezwzględne.

Autorzy \cite{restrictedWeather} zwrócili uwagę na fakt, że użytkownicy robotów domowych przy zadawaniu pytań często pomijają niektóre informacje związne~z lokalizacją i~czasem. Z~tego powodu system wykorzystuje stworzony profil użytkownika by uzupełnić brakujące informacje.

Tak przetworzone dane są wejściem dla klasyfikatora drzewiastego, którego zadaniem jest znalezienie dokładnie do jakiego typu query należy przyporządkować pytanie. Każdemu liściowi drzewa decyzyjnego przypisany jest konkretny schemat zapytania SQL oraz konkretny schemat odpowiedzi. System odrzuci pytanie, jeśli klasyfikator nie znajdzie odpowiedniego query.

Przedstawiony przez autorów \cite{restrictedWeather} system został przetestowany przez 10 osób i~ zostało zadane łacznie 50 pytań. Uzyskano $\num{90.9}$\% poprawnych odpowiedzi, podczas gdy pokrycie wyniosło $\num{75}$\%. Należy jednak zwrócić uwagę na ograniczenia i~sposób działania systemu. Pobiera on dane tylko ze zdefiniowanych przez autorów stron internetowych oraz generowane odpowiedzi są schematyczne~\cite{restrictedWeather}.

\opracowanyartykul{Named entity recognition in a Polish question answering system}
Pozycja \cite{polishQAS} opisuje budowę systemu odpowiadania na pytania w~języku polskim hipisek oraz opisuje wykorzystanie modułu rozpoznawania nazw.

Hipisek jest płytkim systemem odpowiadania na pytania, bazującym na wiedzy pozyskanej z bazy danych artykułów. Autorzy zaimplementowali proste metody bazujące na formułowaniu zapytań internetowych a~następnie wzbogacili system o narzędzie do rozpoznawania nazw Named Entity Recognition (NER). 

System hipisek wykorzystuje algorytm wyszukiwania informacji oraz bazuje na narzędziach lingwistycznych. System zakłada, że odpowiedź na pytanie użytkownika znajduje się bezpośrednio w~dokumencie, dlatego zwracany jest zbiór zdań z~dokumentu.
Zadaniem systemu jest przekształcenie pytania użytkownika do zbioru zapytań do slnika wyszukiwań a~następnie dokonania oceny uzyskanych odpowiedzi i~zwrócenia najlepszej.

Baza danych wykorzystywana przez silnik to zbiór automatycznie generowanych dokumentów tekstowych pochodzących z artykułów z polskich stron internetowych z wiadomościami. Każdy wpis w bazie danych zawiera dodatkowo informacje o tytule, dacie opublikowania, ewentualnych słowach kluczowych oraz adresie url artykułu. Artykuły są indeksowane przy użyciu darmowego silnika indeksowania Sphinx \cite{sphinx}. 

Pierwszym krokiem przetwarzania zapytania jest jest znalezienie bazowej formy wyrazu. W tym celu system wykorzystuje słownik języka polskiego \textit{Dylemat}. Pytanie jest przekształcane w strukturę QQuery zawierającą pozycje:
\begin{itemize}
	\item TOPIC - przedmiot zapytania,
	\item ACTION - główny czasownik w pytaniu,
	\item CONSTRAINTS - ograniczenia, pozostałe leksemy z~pytania zawierające wymagane informacje,
	\item TYPE - pytanie może dotyczyć miejsca, czasu lub osoby.
\end{itemize}
Dodatkowo w strukturze umieszczane są znalezione synonimy bazowych form słów.

W procesie translacji wykorzystywane są dwie metody: 
\begin{itemize}
	\item podejście oparte na zasadach, 
	\item podejście heurystyczne.
\end{itemize}
W celu przekształcenia pytania w formalizm wykorzystane zostało narzędzie Spejd \cite{spejd}. Zasada ta przekształca zdanie w swoistego rodzaju strukturę składającą się ze znanych elementów przy jednoczesnym sprawdzaniu odpowiednich warunków. Jeśli nie udało się przekształcić pytania zykorzystując narzędzie Spejd, system heurystycznie tworzy strukturę bazując na podejściu heurystycznym. Zbudowana struktura QQuery za pomocą metod reformułacji jest przekształcana do formy odpowiedniej dla silnika zapytań.

Formowane są trzy rodzaje zapytań do silkia:
\begin{itemize}
	\item bazowane na frazach,
	\item bazowane na temacie,
	\item zbioru słów z zapytania.
\end{itemize}

Sphinx wykorzystuje schemat Okapi BM25 w celu oceniania jakości dokumentów. System przetwarza dalej tylko najwyżej ocenione dokuemnty.

Każdy dokuemnt otrzymuje dwie oceny: akceptacyjną oraz jakościową. Na oceny wpłwa między innymi: metoda uzyskania dokuemntu, obecność tematu oraz akcji, liczba ograniczeń pytania spełniona przez potencjalną odpowiedź, data publikacji.

Autorzy publikacji \cite{polishQAS} badali jaki wpływ na jakość odpowiedzi będzie miało dodanie do systemu modułu NER. Moduł ten służy do wyszczególniania nazw w tekście i ich klasyfikacji. Bazuje on na darmowych źródłach internetowych. Znalezione nazwy są klasyfikowane do jednej z trzech kategorii: osoba, miejsce lub czas.

Stworzony moduł NER pozwolił na poprawę jakości systemu hipisek o około $40$\%. Autorzy porównywali uzyskane wyniki z system Ktoco.pl oraz z systemem Hipisek (bez modułu NER). Wykorzystano w tym celu test regresyjny bazujący na pomyśle wykorzystywanym w konkursie TREC.
