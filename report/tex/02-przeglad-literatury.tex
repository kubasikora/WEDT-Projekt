\todo{Opisać swoje artykuły do 8 kwietnia, nie zapominajmy o bibliografi i cytowaniu} 

\opracowanyartykul{answerbus}
AnswerBus~\cite{zheng2002answerbus} jest systemem odpowiadającym na pytania ogólne, bazującym na wyszkuiwaniu informacji, opartym o~analizę zdań. System przyjmuje pytania w~sześciu językach (angielskim, niemieckim, francuskim, hiszpańskim, włoskim i~portugalskim) oraz odpowiada po angielsku. Odpowiedzi wyszukuje z~internetu, posiłkując się pięcioma różnymi wyszukiwarkami internetowymi m.in. Google czy AltaVista.

Przedstawione rozwiązanie składa się z~kilku komponentów, przetwarzających liniowo otrzymane pytanie. W pierwszej kolejności pytanie trafia do modułu rozpoznającego język. Jeżeli pytanie jest po angielsku to trafia ono dalej, natomiast jeżeli nie, to zostaje ono przetłumaczone wykorzystując internetową usługę tłumacza. Pytanie w~języku angielskim jest następnie analizowane, w~celu określenia typu pytania. Na podstawie typu, wybierane są trzy z~pięciu najbardziej odpowiednich przeglądarek. Wybrane rozwiązania są następnie odpytywane o~specjalnie przygotowane zapytanie, a~następnie z~otrzymanych dokumentów tekstowych wybierane są zdania, potencjalnie zawierające odpowiedź na pytanie. Ostatnim krokiem jest ocena wybranych zdań i~zwrócenie najlepszych wyników użytkownikowi~\cite{zheng2002answerbus}.  

System AnswerBus, podobnie jak większość tego typu rozwiązań, określa typ postawionego pytania. Oprócz oczywistych typów, na przykład w~pytaniu \emph{Jak daleko}, określa typ \emph{odległość}, system stara się ocenić dodatkowe parametry typu oczekiwany rząd wielkości (w~odpowiedzi na przykładowe pytanie oczekujemy odległości w~kilometrach a~nie w~centymetrach). Stwierdzenie typu pytania, pozwala na dokładniejszy dobór wyszukiwarki, od bardzo ogólnej jak Google do specjalistycznych jak Yahoo News~\cite{zheng2002answerbus}.

Analiza odpowiedzi sprowadza się do oceny otrzymanego zdania na podstawie treści pytania. Oczekiwana odpowiedź powinna zawierać jak najwięcej wyrazów z~zapytania. Do oceny odpowiedzi, nie są brane pod uwagę wyrazy nie przenoszące szczególnej informacji typu przyimki, zaimki, spójniki czy okoliczniki. Do oceny odpowiedzi stosuje wzór~\ref{eqn:ocena-answerbus},gdzie $q$ to liczba pasujących słów pytania a~$Q$ to liczba wszystkich wyrazów w~odpowiedzi~\cite{zheng2002answerbus}.

\begin{equation}
    \label{eqn:ocena-answerbus}
    q \geq  \sqrt{Q - 1}  + 1
\end{equation}

Oprócz zliczania słów, system wykrywa powiązania pomiędzy rzeczownikami a~zaimkami i~w~momencie wykrycia zaimka, stara się dokonać podstawienia bazując na zdaniu poprzednim. Na ocenę wpływają również obecność synonimów słów kluczowych oraz pozycja wyniku w~wyszukiwarce~\cite{zheng2002answerbus}.

\opracowanyartykul{asmr}
System AskMSR~\cite{brill2002analysis} jest systemem zajmującym się odpowiadaniem na pytania ogólne. W~tym celu przeszukuje on sieć WWW, wykorzystując internetowe wyszukiwarki. Celem autorów w~tym projekcie, było stworzenie systemu odpowiadającego na pytania, bez przeprowadzania zaawansowanej analizy lingwistycznej zarówno pytania,jak i~potenjalnych odpowiedzi. Głównym założeniem był fakt olbrzymiej redundancji odpowiedzi w~sieci WWW.

Zasadniczo, system składa się z~czterech modułów. Pierwszy z~nich na podstawie pytania buduje zapytania do wyszukiwarki internetowej. Każde z~zapytań, zawiera pewną część oryginalnego pytania. Im więcej wyrazów oryginalnego zapytania znajduje się w~przygotowanym zapytaniu, tym z~większą wagą będą brane pod uwagę odpowiedzi na nie otrzymane.Po konstrukcji zapytań, są one wysyłane do wyszukiwarki internetowej. Do dalszych modułów wysyłane są jedynie krótkie wyrywki dokumentu, przesyłane wraz z~linkami~\cite{brill2002analysis}.

Drugim komponent systemu AskMSR zajmuje się wyszukiwaniem \emph{n-gramów} w~zwróconych odpowiedziach. W~tekstach wyszukiwane są tylko unigramy, bigramy oraz trigramy. Każdy z~\emph{n-gramów} jest oceniany na podstawie wcześniejszej oceny zapytania. Do oceny wliczana jest również liczba unikalnych odpowiedzi zwróconych przez system zawierający dany \emph{n-gram}~\cite{brill2002analysis}.

Przygotowane \emph{n-gramy} są przekazywane do trzeciego modułu filtrowania. Moduł filtrowania, na podstawie ręcznie przygotowanych filtrów opartych o~wyrażenia regularne, określa jeden z~siedmiu typów pytania, a~następnie filtruje \emph{n-gramy}, zawężając potencjalny zestaw odpowiedzi. Ostatnim elementem systemu jest moduł łączenia, który wiąże ze sobą nakładające się \emph{n-gramy}, sukcesywnie budując większe zdania, normalnie niemożliwe do zbudowania wyłącznie poprzez tworzenie \emph{n-gramów}. Ocena połączonych \emph{n-gramów} jest określana na podstawie wyżej ocenianego składnika. Łączenie odbywa się do momentu aż nie można łączyć żadynch odpowiedzi~\cite{brill2002analysis}.

Autorzy systemu zwracają uwagę że najwiekszy wpływ na jakość zwracanych odpowiedzi mają wpływ moduły filtrowania oraz łączenia \emph{n-gramów}. System został przetestowany przy pomocy pytań TREC-9~\cite{voorhees2001trec}. Najbardziej problematyczne okazały się pytania rozpoczynające się od \emph{jak?}, natomiast najlepiej radził sobie z~pytaniami rozpoczynającymi się od \emph{kto?}, z~dużą liczbą typowych dla języka słów~\cite{brill2002analysis}.