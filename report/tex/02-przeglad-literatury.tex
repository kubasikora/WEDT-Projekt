\todo{Opisać swoje artykuły do 8 kwietnia, nie zapominajmy o bibliografi i cytowaniu} 

\opracowanyartykul{answerbus}
AnswerBus~\cite{zheng2002answerbus} jest systemem odpowiadającym na pytania ogólne, bazującym na wyszkuiwaniu informacji, opartym o~analizę zdań. System przyjmuje pytania w~sześciu językach (angielskim, niemieckim, francuskim, hiszpańskim, włoskim i~portugalskim) oraz odpowiada po angielsku. Odpowiedzi wyszukuje z~internetu, posiłkując się pięcioma różnymi wyszukiwarkami internetowymi m.in. Google czy AltaVista.

Przedstawione rozwiązanie składa się z~kilku komponentów, przetwarzających liniowo otrzymane pytanie. W pierwszej kolejności pytanie trafia do modułu rozpoznającego język. Jeżeli pytanie jest po angielsku to trafia ono dalej, natomiast jeżeli nie, to zostaje ono przetłumaczone wykorzystując internetową usługę tłumacza. Pytanie w~języku angielskim jest następnie analizowane, w~celu określenia typu pytania. Na podstawie typu, wybierane są trzy z~pięciu najbardziej odpowiednich przeglądarek. Wybrane rozwiązania są następnie odpytywane o~specjalnie przygotowane zapytanie, a~następnie z~otrzymanych dokumentów tekstowych wybierane są zdania, potencjalnie zawierające odpowiedź na pytanie. Ostatnim krokiem jest ocena wybranych zdań i~zwrócenie najlepszych wyników użytkownikowi~\cite{zheng2002answerbus}.  

System AnswerBus, podobnie jak większość tego typu rozwiązań, określa typ postawionego pytania. Oprócz oczywistych typów, na przykład w~pytaniu \emph{Jak daleko}, określa typ \emph{odległość}, system stara się ocenić dodatkowe parametry typu oczekiwany rząd wielkości (w~odpowiedzi na przykładowe pytanie oczekujemy odległości w~kilometrach a~nie w~centymetrach). Stwierdzenie typu pytania, pozwala na dokładniejszy dobór wyszukiwarki, od bardzo ogólnej jak Google do specjalistycznych jak Yahoo News~\cite{zheng2002answerbus}.

Analiza odpowiedzi sprowadza się do oceny otrzymanego zdania na podstawie treści pytania. Oczekiwana odpowiedź powinna zawierać jak najwięcej wyrazów z~zapytania. Do oceny odpowiedzi, nie są brane pod uwagę wyrazy nie przenoszące szczególnej informacji typu przyimki, zaimki, spójniki czy okoliczniki. Do oceny odpowiedzi stosuje wzór~\ref{eqn:ocena-answerbus},gdzie $q$ to liczba pasujących słów pytania a~$Q$ to liczba wszystkich wyrazów w~odpowiedzi~\cite{zheng2002answerbus}.

\begin{equation}
    \label{eqn:ocena-answerbus}
    q \geq  \sqrt{Q - 1}  + 1
\end{equation}

Oprócz zliczania słów, system wykrywa powiązania pomiędzy rzeczownikami a~zaimkami i~w~momencie wykrycia zaimka, stara się dokonać podstawienia bazując na zdaniu poprzednim. Na ocenę wpływają również obecność synonimów słów kluczowych oraz pozycja wyniku w~wyszukiwarce~\cite{zheng2002answerbus}.

\opracowanyartykul{asmr}\label{askmr}
System AskMSR~\cite{brill2002analysis} jest systemem zajmującym się odpowiadaniem na pytania ogólne. W~tym celu przeszukuje on sieć WWW, wykorzystując internetowe wyszukiwarki. Celem autorów w~tym projekcie, było stworzenie systemu odpowiadającego na pytania, bez przeprowadzania zaawansowanej analizy lingwistycznej zarówno pytania,jak i~potenjalnych odpowiedzi. Głównym założeniem był fakt olbrzymiej redundancji odpowiedzi w~sieci WWW.

Zasadniczo, system składa się z~czterech modułów. Pierwszy z~nich na podstawie pytania buduje zapytania do wyszukiwarki internetowej. Każde z~zapytań, zawiera pewną część oryginalnego pytania. Im więcej wyrazów oryginalnego zapytania znajduje się w~przygotowanym zapytaniu, tym z~większą wagą będą brane pod uwagę odpowiedzi na nie otrzymane.Po konstrukcji zapytań, są one wysyłane do wyszukiwarki internetowej. Do dalszych modułów wysyłane są jedynie krótkie wyrywki dokumentu, przesyłane wraz z~linkami~\cite{brill2002analysis}.

Drugim komponent systemu AskMSR zajmuje się wyszukiwaniem \emph{n-gramów} w~zwróconych odpowiedziach. W~tekstach wyszukiwane są tylko unigramy, bigramy oraz trigramy. Każdy z~\emph{n-gramów} jest oceniany na podstawie wcześniejszej oceny zapytania. Do oceny wliczana jest również liczba unikalnych odpowiedzi zwróconych przez system zawierający dany \emph{n-gram}~\cite{brill2002analysis}.

Przygotowane \emph{n-gramy} są przekazywane do trzeciego modułu filtrowania. Moduł filtrowania, na podstawie ręcznie przygotowanych filtrów opartych o~wyrażenia regularne, określa jeden z~siedmiu typów pytania, a~następnie filtruje \emph{n-gramy}, zawężając potencjalny zestaw odpowiedzi. Ostatnim elementem systemu jest moduł łączenia, który wiąże ze sobą nakładające się \emph{n-gramy}, sukcesywnie budując większe zdania, normalnie niemożliwe do zbudowania wyłącznie poprzez tworzenie \emph{n-gramów}. Ocena połączonych \emph{n-gramów} jest określana na podstawie wyżej ocenianego składnika. Łączenie odbywa się do momentu aż nie można łączyć żadynch odpowiedzi~\cite{brill2002analysis}.

Autorzy systemu zwracają uwagę że najwiekszy wpływ na jakość zwracanych odpowiedzi mają wpływ moduły filtrowania oraz łączenia \emph{n-gramów}. System został przetestowany przy pomocy pytań TREC-9~\cite{voorhees2001trec}. Najbardziej problematyczne okazały się pytania rozpoczynające się od \emph{jak?}, natomiast najlepiej radził sobie z~pytaniami rozpoczynającymi się od \emph{kto?}, z~dużą liczbą typowych dla języka słów~\cite{brill2002analysis}.

\opracowanyartykul{question answering systems ibm watson}
W pracy~\cite{lapshin2012question} przedstawiona została ogólna struktura systemu do odpowiadania na pytania ogólne. Autor podzielił proces odpowiadania na kilka odrębnych faz:
\begin{itemize}
    \item określanie typu pytania,
    \item przetwarzanie pytania,
    \item określenie kontekstu,
    \item przeszukiwanie bazy wiedzy,
    \item konstruowanie odpowiedzi.
\end{itemize}

Oprócz tego, w~pracy zostały określone obostrzenia, które powinien spełniać system:
\begin{itemize}
    \item interaktywność,
    \item odpowiadanie w~czasie rzeczywistym,
    \item logiczne wnioskowanie,
    \item profilowanie użytkownika,
    \item odpowiadanie w~różnych językach.
\end{itemize}

Przykładowym systemem, spełniającym przedstawione wymagania, jest \emph{DeepQA}, znany obecnie pod nazwą \emph{IBM Watson}. W~pierwszym etapie, system analizuje pytania, określając jego klasę, typ oczekiwanej odpowiedzi oraz buduje związki logiczne pomiędzy wyrazami w~pytaniu. Dzięki temu, jest on w~stanie zbudować odpowiednie zapytanie do bazy wiedzy, którą w~tym przypadku jest sieć WWW. W~kolejnym kroku, \emph{Watson} tworzy hipotezy odpowiedzi, które następnie sprawdza, poszukując dowodów oraz dokonując analizy krytycznej źródeł. Ostatecznie, ze zbioru odpowiedzi wybierana jest ta najbardziej prawdopodobna i~to ona zostaje zwrócona użytkownikowi~\cite{lapshin2012question}.

Najważniejszym elementem całego systemu jest moduł klasyfikacji odpowiedzi. Poprawna klasyfikacja jest odpowiedzialna za największy przyrost poprawnych odpowiedzi. System korzysta ze zbioru kategorii pytań, nazywany ontologią. Pierwsze zbiory określały od 13 do 18 kategorii pytań, natomiast w~późniejszych pracach, zaczęto dzielić kategorię na kategorię główne i~szczegółowe, przykładowo w~\cite{li2002learning} określono sześć typów głównych i prawie pięćdziesiąt typów szczegółowych. Do określania typu pytania najczęściej wykorzystywane są klasyfikatory regułowe oraz statystyczne. Tego typu naiwne podejście jest bardzo proste i~daje dobre rezultaty~\cite{lapshin2012question}.

To co pozwala dobrze oceniać system \emph{IBM Watson} to nie tylko dobre wyniki w~testach na pytaniach z~konferencji TREC, ale również potężny sukces biznesowy, który firma IBM osiągnęła sprzedając swoje rozwiązanie\footnote{https://www.ibm.com/watson}. W~połączeniu z~otwartą chmurą, nowoczesne przedsiębiorstwa mogą wykorzystać Watsona do tworzenia inteligentnych systemów zarządzania oraz wspierania ludzi w~przemyśle, medycynie i~logistyce. Najbardziej znanym przykładem użycia systemu \emph{IBM Watson} jest jego udział w~programie \emph{Jeopardy!}, w~którym to uczestnicy odpowiadają na zróżnicowane i~skomplikowane pytania~\cite{lapshin2012question}. 


\opracowanyartykul{A Practical QA System in Restricted Domains}
Praca ~\cite{restrictedWeather} przedstawia opis systemu odpowiadającego na pytania stworzonego dla robota pracującego w~warunkach domowych. Autorzy zauważyli, że systemy tego typu są wykorzystywane przeważnie w~ograniczonych dziedzinach takich jak pogoda czy program stacji telewizyjnych oraz, że dla użytkowników takich systemów znacznie ważniejsza jest precyzja uzyskanych odpowiedzni. Artykuł skupił swoją uwagę na opisie systemu umożliwiającemy zadawanie pytań na temat pogody. 

Proces odpowiadania na pytania jest trójfazowy: 
\begin{itemize}
	\item przetworzenie zapytania w języku naturalnym na zapytanie SQL,
	\item realizacja zapytania,
	\item konwersja otrzymanego rezultatu na język naturalny.
\end{itemize}
Stworzony system składa się z trzech elementów: 
\begin{itemize}
	\item silnika IE, 
	\item DBMS, 
	\item silnika QA.
\end{itemize}

Dane wykorzystywane do odpowiedzi na pytania pochodzą z~wybranych przez autorów, połowicznie ustrukuryzowanych stron internetowych Korea Meteorological Administration. Moduł silnika IE co godzinę pobiera z~odpowiedniej strony internetowej najnowsze dokumenty, parsuje je w~celu wydobycia zdyskretyzowanych informacji i zapisuje je w bazie danych. 

Silnik QA służy do analizowania zapytań oraz konwertowania uzyskanych odpowiedzi na język naturalny. Autorzy \cite{restrictedWeather} wydzielili skończoną liczbę tematów (związanych z pogodą) o~jakie może zapytać użytkownik. Są to między innymi: temperatura powietrza, prędkość wiatru, opady. Analiza pytania polega na ekstrakcji słów kluczowych dotyczących m.in.: tematu pytania, miejsca, czasu. Aby móc zidentyfikować dokładnie intencje człowieka, przy użyciu specjalnego modułu, wyszczególnione z~zapytania słowa kluczowe są konwertowane na konretne terminy stosowane w systemie. Dzięki temu pytanie o konieczność wzięcia parasola jest traktowane tak samo jak pytanie czy pada aktualnie deszcz. W~kolejnym etapie przetwarzania pytania system zamienia wyrażenia czasowe takie jak \textit{dzisiaj} i~\textit{jutro} na wartości bezwzględne.

Autorzy \cite{restrictedWeather} zwrócili uwagę na fakt, że użytkownicy robotów domowych przy zadawaniu pytań często pomijają niektóre informacje związne~z lokalizacją i~czasem. Z~tego powodu system wykorzystuje stworzony profil użytkownika by uzupełnić brakujące informacje.

Tak przetworzone dane są wejściem dla klasyfikatora drzewiastego, którego zadaniem jest znalezienie dokładnie do jakiego typu query należy przyporządkować pytanie. Każdemu liściowi drzewa decyzyjnego przypisany jest konkretny schemat zapytania SQL oraz konkretny schemat odpowiedzi. System odrzuci pytanie, jeśli klasyfikator nie znajdzie odpowiedniego query.

Przedstawiony przez autorów \cite{restrictedWeather} system został przetestowany przez 10 osób i~ zostało zadane łacznie 50 pytań. Uzyskano $\num{90.9}$\% poprawnych odpowiedzi, podczas gdy pokrycie wyniosło $\num{75}$\%. Należy jednak zwrócić uwagę na ograniczenia i~sposób działania systemu. Pobiera on dane tylko ze zdefiniowanych przez autorów stron internetowych oraz generowane odpowiedzi są schematyczne~\cite{restrictedWeather}.

\opracowanyartykul{Named entity recognition in a Polish question answering system}
Pozycja \cite{polishQAS} opisuje budowę systemu odpowiadania na pytania w~języku polskim hipisek oraz opisuje wykorzystanie modułu rozpoznawania nazw.

Hipisek jest płytkim systemem odpowiadania na pytania, bazującym na wiedzy pozyskanej z bazy danych artykułów. Autorzy zaimplementowali proste metody bazujące na formułowaniu zapytań internetowych a~następnie wzbogacili system o narzędzie do rozpoznawania nazw Named Entity Recognition (NER). 

System hipisek wykorzystuje algorytm wyszukiwania informacji oraz bazuje na narzędziach lingwistycznych. System zakłada, że odpowiedź na pytanie użytkownika znajduje się bezpośrednio w~dokumencie, dlatego zwracany jest zbiór zdań z~dokumentu.
Zadaniem systemu jest przekształcenie pytania użytkownika do zbioru zapytań do slnika wyszukiwań a~następnie dokonania oceny uzyskanych odpowiedzi i~zwrócenia najlepszej.

Baza danych wykorzystywana przez silnik to zbiór automatycznie generowanych dokumentów tekstowych pochodzących z artykułów z polskich stron internetowych z wiadomościami. Każdy wpis w bazie danych zawiera dodatkowo informacje o tytule, dacie opublikowania, ewentualnych słowach kluczowych oraz adresie url artykułu. Artykuły są indeksowane przy użyciu mechanizmu indeksowania w darmowym silniku zapytań Sphinx \cite{sphinx}. 

Pierwszym krokiem przetwarzania zapytania jest jest znalezienie bazowej formy wyrazu. W tym celu system wykorzystuje słownik języka polskiego \textit{Dylemat}. Pytanie jest przekształcane w strukturę QQuery zawierającą pozycje:
\begin{itemize}
	\item TOPIC - przedmiot zapytania,
	\item ACTION - główny czasownik w pytaniu,
	\item CONSTRAINTS - ograniczenia, pozostałe leksemy z~pytania zawierające wymagane informacje,
	\item TYPE - pytanie może dotyczyć miejsca, czasu lub osoby.
\end{itemize}
Dodatkowo w strukturze umieszczane są znalezione synonimy bazowych form słów.

W procesie translacji wykorzystywane są dwie metody: 
\begin{itemize}
	\item podejście oparte na zasadach, 
	\item podejście heurystyczne.
\end{itemize}
W celu przekształcenia pytania w formalizm wykorzystane zostało narzędzie Spejd \cite{spejd}. Zasada ta przekształca zdanie w swoistego rodzaju strukturę składającą się ze znanych elementów przy jednoczesnym sprawdzaniu odpowiednich warunków. Jeśli nie udało się przekształcić pytania wykorzystując narzędzie Spejd, system tworzy strukturę bazując na podejściu heurystycznym. Zbudowana struktura QQuery za pomocą metod reformułacji jest przekształcana do formy odpowiedniej do silnika zapytań.

Formowane są trzy rodzaje zapytań:
\begin{itemize}
	\item bazowane na frazach,
	\item bazowane na temacie,
	\item zbioru słów z zapytania.
\end{itemize}

Sphinx wykorzystuje schemat Okapi BM25 w celu oceniania jakości dokumentów. System przetwarza dalej tylko najwyżej ocenione dokumenty.

Każdy dokuemnt otrzymuje dwie oceny: akceptacyjną oraz jakościową. Na oceny wpłwa między innymi: metoda uzyskania dokuemntu, obecność tematu oraz akcji, liczba ograniczeń pytania spełniona przez potencjalną odpowiedź, data publikacji.

Autorzy publikacji \cite{polishQAS} badali jaki wpływ na jakość odpowiedzi będzie miało dodanie do systemu modułu NER. Moduł ten służy do wyszczególniania nazw w tekście i ich klasyfikacji. Bazuje on na darmowych źródłach internetowych ze zbiorami nazw. Znalezione nazwy są klasyfikowane do jednej z trzech kategorii: osoba, miejsce lub czas.

Stworzony moduł NER pozwolił na poprawę jakości systemu hipisek o około $40$\%. Autorzy porównywali uzyskane wyniki z system Ktoco.pl oraz z systemem Hipisek (bez modułu NER). Wykorzystano w tym celu test regresyjny bazujący na pomyśle wykorzystywanym w konkursie TREC \cite{polishQAS}.

\opracowanyartykul{Data-Intensive Question Answering}
Pozycja \cite{brill2001data} opisuje wykorzystanie systemu AskMSR oraz AskMSR2 na konferencji TREC-9 w 2001 roku. Jako że analiza systemów została przedstawiona w punkcie \ref{askmr} tego rozdziału (\cite{brill2002analysis}), w opisie artykułu pominięte zostaną kwestie poruszone powyżej.

Główną modyfikacją systemów AskMSR oraz AskMSR na potrzeby konferencji TREC było dodanie fazy sprawdzającej wystąpnie potencjalnej odpowiedzi w zbiorze dokumentów TREC. System po znalezieniu n najlepszych odpowierdzi z zasobów internetowych, wyszukiwał ich w bazie danych TREC-9. W tym celu wykorzystany został system Okapi IR. Zapytanie polegało na stworzeniu query skłądającego się ze słów użytych w potencjalnej odpowiedzi. Następnie znalezione dokuemnty zostały ocenione.

Autorzy \cite{brill2001data} na tym etapie generacji odpowiedzi nie wykorzystywali specjalnych zabiegów lingwistycznych w celu zwiększenia precyzji ani pokrycia. Do dalszego przetwarzania wybierane były najwyżej ocenione dokumenty dla każdej potencjalnej odpowiedzi.

Ostateczne odpowiedzi systemu były w większości przypadków generowane na podstawie najwyżej ocenianych dokumentów wspierających potencjalne odpowiedzi. Wyjątkiem od reguły była sytuacji, gdzie jeden dokuemnt powtarzał się w kilku potencjalnych odpowiedziach.

Autorzy \cite{brill2001data} zwracają uwagę, że mechanizm ten może być wykorzystywany także w innych przypadkach. Wskazują, że można go wykorzystać do weryfikacji potencjalnej odpowiedzi znalezionej w internecie w niewielkich, ale wiarygodnych zbiorach takich jak słowniki lub bazy arykułów \cite{brill2001data}.

Stworzony moduł NER pozwolił na poprawę jakości systemu hipisek o około $40$\%. Autorzy porównywali uzyskane wyniki z system Ktoco.pl oraz z systemem Hipisek (bez modułu NER). Wykorzystano w tym celu test regresyjny bazujący na pomyśle wykorzystywanym na konferencjach TREC \cite{brill2001data}.

\opracowanyartykul{issues of polish QA~\cite{przybyla2012issues}}
W~pracy~\cite{przybyla2012issues}, autor opisuje budowę typowego systemu do odpowiadania na pytania otwarte, porównując które elementy takiego rozwiązania są zależne od języka. Oprócz tego, podkreślone zostały cechy języka polskiego, jako języka słowiańskiego, które utrudniają automatyczne przetwarzanie oraz generację tekstu.

Główną cechą czyniącą przetwarzanie tekstów w~języku polskim trudnym, jest mnogość form językowych. Fleksyjność języków słowiańskich pozwala na określanie roli wyrazu w~zdaniu na podstawie końcówki, a~nie jak w~języku angielskim na podstawie pozycji w~zdaniu. Przykładowo zdanie \emph{Jan zjadł rybę} oraz \emph{Rybę zjadł Jan}, znaczą to samo, mimo inaczej położonego akcentu, natomiast zdania \emph{John ate a~fish} i~\emph{A~fish ate John} znaczą zupełnie co innego. Co więcej, złożona odmiana rzeczowników sprawia że rozpoznawanie nazw własnych jest jeszcze trudniejsze~\cite{przybyla2012issues}.

Jako komponenty zależne od języka, autor wydzielił:
\begin{itemize}
	\item przetwarzanie bazy wiedzy,
	\item przetwarzanie pytania,
	\item generacja odpowiedzi,
	\item rozwijanie anafor.
\end{itemize}

W~przypadku przetwarzania bazy wiedzy, autor podkreśla że klasyczny \emph{stemming} nie zdaje egzaminu oraz że o~wiele lepsze wyniki uzyskuje się przy użyciu pełnych analizatorów morfologicznych, które przetwarzają tekst wolniej jednak gwarantują znalezienie każdej formy językowej dla przetwarzanego segmentu.

Mimo wielu różnic, autor podkreśla że kilka operacji, skutecznie używanych w~systemach do odpowiadania w~języku angielskim, może sprawdzić się w~systemach obsługujących język polski. Taką operacją jest między innymi indeksowanie bazy wiedzy, w~celu przyspieszenia wyszukiwania tekstu oraz wyszukiwanie relacji pomiędzy wyrazami, wykorzystując narzędzia typu \emph{WordNet}.

\opracowanyartykul{A survey of Text Question Answering Techniques~\cite{gupta2012survey}}
Według~\cite{gupta2012survey}, proces automatycznego odpowiadania na pytania można podzielić na trzy oddzielne fazy: klasyfikację pytania, wydobycie informacji z~bazy wiedzy oraz wydobycie/generacja odpowiedzi. Te trzy fazy są wyróżnialne w~każdym systemie do odpowiadania na pytania, niezależnie od wykorzystanego mechanizmu. Autorzy wyodrębnili cztery różne podejścia do rozwiązania postawionego problemu, wyróżniając systemy oparte~o:
\begin{itemize}
	\item przeszukiwanie sieci WWW,
	\item wyszukiwanie w~bazie dokumentów tekstowych,
	\item wyszukiwanie w~zamkniętej bazie danych,
	\item reguły i~dopasowywanie do wzorca.
\end{itemize}

Autorzy podkreślają jak ważne jest automatyczne wydobycie informacji z~przygotowanego dokumentu. W~celu poprawnego wydobycia poszukiwanych informacji z~dokumentu należy w~pierwszej kolejności dokonać filtrowania akapitów. Zazwyczaj dokonuje się tego wykorzystując dopasowywanie do słów kluczowych zapytania. Im więcej razy w~tekście wystąpi słowo kluczowe, tym większa szansa na to że nie zostanie ono odfiltrowane. Następnie, każdemu pozostałemu akapitowi należy przyporządkować liczbę, ocenę jakości tekstu.
Dobór funkcji oceniającej jest zależny od typu przetwarzanych dokumentów oraz typu pytania. Ocenione akapity należy posortować a~następnie, rozpoczynając od pierwszego, przeprowadzić proces leksykalnego i~syntaktycznego wydobycia wiedzy z~tekstu~\cite{gupta2012survey}.

Aby odpowiednio ocenić akapity, należy przed przetwarzaniem dokumentów dokonać klasyfikacji pytania. Autorzy~\cite{gupta2012survey} wyodrębnili osiem klas głównych pytań odpowiadających na pytania:
\begin{itemize}
	\item co?,
	\item jak?,
	\item dlaczego?,
	\item który/która/które/którzy?,
	\item gdzie?,
	\item czyji/kogo?
	\item kiedy?,
	\item pytania funkcjonalne.
\end{itemize}

Pierwsze siedem klas pytań jest łatwo rozróżnialnych, za pomocą stosownych reguł logicznych i~wyrażeń regularnych. Największą trudność sprawiają pytania funkcjonalne typu \emph{Nazwij piłkarza który strzelił jedyną bramkę w~finale Euro 2016}. Pytania tego typu wymagają szczególnego traktowania i~najczęściej wymagają maszynowego rozumienia tekstu.

\opracowanyartykul{Data-Intensive Question Answering}
Pozycja \cite{brill2001data} opisuje wykorzystanie systemu AskMSR oraz AskMSR2 na konferencji TREC-9 w~2001 roku. Jako że analiza systemów została przedstawiona w~punkcie \ref{askmr} tego rozdziału (\cite{brill2002analysis}), w~opisie artykułu pominięte zostaną kwestie poruszone powyżej.

Główną modyfikacją systemów AskMSR oraz AskMSR na potrzeby konferencji TREC było dodanie fazy sprawdzającej wystąpnie potencjalnej odpowiedzi w zbiorze dokumentów TREC. System po znalezieniu n~najlepszych odpowierdzi z~zasobów internetowych, wyszukiwał ich w~bazie danych TREC-9. W~tym celu wykorzystany został system Okapi IR. Zapytanie polegało na stworzeniu query skłądającego się ze słów użytych w~potencjalnej odpowiedzi. Następnie znalezione dokuemnty zostały ocenione.

Autorzy \cite{brill2001data} na tym etapie generacji odpowiedzi nie wykorzystywali specjalnych zabiegów lingwistycznych w~celu zwiększenia precyzji ani pokrycia. Do dalszego przetwarzania wybierane były najwyżej ocenione dokumenty dla każdej potencjalnej odpowiedzi.

Ostateczne odpowiedzi systemu były w większości przypadków generowane na podstawie najwyżej ocenianych dokumentów wspierających potencjalne odpowiedzi. Wyjątkiem od reguły była sytuacji, gdzie jeden dokuemnt powtarzał się w~kilku potencjalnych odpowiedziach.

Autorzy \cite{brill2001data} zwracają uwagę, że mechanizm ten może być wykorzystywany także w~innych przypadkach. Wskazują, że można go wykorzystać do weryfikacji potencjalnej odpowiedzi znalezionej w~internecie w~niewielkich, ale wiarygodnych zbiorach takich jak słowniki lub bazy arykułów \cite{brill2001data}.

\opracowanyartykul{An English language question answering system for a large relational database}
Pozycja \cite{waltz1978english} opisuje sposób działania oraz budowę systemu odpowiadania na pytania PLANES. System ten został stworzony w celu zapewnienia wygodnego interfejsu do bazy danych związanej z samolotami dla amerykańskiego wojska w drugiej połowie lat 70-tych ubiegłego wieku.

Artykuł \cite{waltz1978english} opisuje budowę całego systemu, ale z perspektywy tworzenia nowego systemu istotne są tylko kwestie związane z konstrukcją parsera. Przetwarzanie zapytania składa się z czterech etapów: parsowania pytania, konstrukcji odpowiedniego query, dokonania jego oceny oraz wykonania zapytania i zwrócenia odpowiedzi. Założenia z jakich wyszli autorzy systemu to: wykorzystywanie specyficznego słownictwa przez użytkowników a co za tym idzie brak dwuznaczności oraz fakt, że użytkownicy systemu nie wpisują długich zapytań.

System podzielił parsowanie pytania na kilka etapów. W pierwszym dokonywana była poprawa literówek oraz konwersja słów do ich kanoniczneg postaci. W drugim etapie wykorzystywane były podsieci ATN, których zadaniem było połączenie frazy ze zdefiniowanym w systemie znaczeniem. Dla każdego rodzaju obiektu (na przykład czas, rodzaje samolotów, rodzaje usterek) zaimplementowana została osobna podsieć identyfikująca znaczenie. 

Zaletą systemu było szybkie odrzucanie słów nie niosących żadnej informacji, które przez autorów \cite{waltz1978english} zostały nazwane szumami. Identyfikowane one były za pomocą osobnej podsieci. 

System, jeśli udało mu się znaleźć znaczenie jednego słowa używając danej podsieci, starał się wykorzystać ją ponownie w procesie identyfikacji kolejnego słowa.

Po przeanalizowaniu wszystkich słów oraz odrzuceniu szumów system aktualizował rejestr kontekstu, który został zaimplementowany w postaci stosu. W rejestrze tym znajdują się ostatie zapytania wraz z ostatnio wykrytymi frazami, ostatnio użytymi query oraz znalezionymi odpowiedziami. Do rejestru zostały wrzucane w tej fazie parsowania kanoniczne formy słów wraz z ich znaczeniem. Rejestr ten gra istotną rolę w procesie ustalania niesprecyzowanych zmiennych. Służy między innymi do znajdowania właściwych znaczeń zaimków.

Jeśli system nie jest w stanie jednoznaczie odtworzyć znaczenia danego słowa, to tworzone są struktury zwane ramkami pojęć (ang. concept case frames). Struktura składa się z czasownika oraz części rzeczownikowych, do każdej przypisany jest wzór query. System testuje struktury oraz sprawdza, czy ich wykorzystanie może mieć sens. Jeśli wynik testowania jest pozytywny dla większej niż jednej struktury, system prosi użytkownika o wybranie znaczenia oraz zwraca znalezione ramki.

Tak stworzone struktury przekzywane są następnie do modułu tworzącego zapytania. Proces tworzenia zapytania rozpoczyna się od próby identyfikacji tablic potrzebnych do znalezienia danych. Następnie moduł sprawca jakiej odpowiedzi oczekuje użytkownik, czyli jakiej domeny powinien być zwrócony wynik. Jako że system był tworzony w latach 70-tych, dużym wyzwaniem było przetważanie złożonych zapytań.

Przed rozpoczęciem realizacji zapytania, PLANES przekazywał użytkownikowi stworzone query oraz umożliwił wprowadzanie korekt. Dane zwrócone z bazy danych były przekazywane do generatora odpowiedzi, który na podstawie rodzaju zwróconych danych oraz polecenia użytkownika konwertował je do postaci grafu lub tabeli\cite{waltz1978english}.

