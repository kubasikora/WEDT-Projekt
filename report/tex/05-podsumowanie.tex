W ramach etapu przeprowadzone zostały studia literaturowe na temat systemów odpowiadania na pytania. Na ich podstawie określono zakres projektu, tj. zaprojektowanie i implementacja własnego systemu odpowiadającego na pytania ogólne o określonych typach odpowiedzi (osoba, miejsce, data, cechy wielkościowe, przedmiot) zadane w języku polskim. W raporcie przedstawiono projekt rozwiązania oraz zarys algorytmu odpowiadania na pytania.  Na ich podstawie dokonana została dekompozycja tworzonego systemu. Wyszczególnione zostały komponenty do samodzielnej implementacji, a także wskazano, z jakich gotowych modułów będzie składał się system. Na bazie wydzielonych komponentów przedstawiona została propozycja podziału systemu na aplikację. Na koniec opisano przygotowane dane testowe, zawierające 250 pytań i odpowiedzi.


Stworzona koncepcja systemu umożliwia jego skalowalność w przyszłości. Po zakończeniu projektu można będzie go dalej rozwijać poprzez dodawanie kolejnych obsługiwanych typów odpowiedzi, a także ulepszać przez wykorzystywane nowych modułów zewnętrznych.  Decyzja o wyborze języka polskiego jako język zadawania pytań i udzielania odpowiedzi zwiększa, naszym zdaniem, atrakcyjność systemu, z uwagi na niewielką ilość dostępnych rozwiązań w tym języku. Dodatkowym walorem dla nas jest także sposobność przetestowania istniejących narzędzi dla języka polskiego.