\documentclass{article}
\pdfpagewidth=8.5in
\pdfpageheight=11in

\usepackage{WEDTreport}
% Use the postscript times font!
\usepackage{times}
\usepackage{soul}
\usepackage{url}
\usepackage{xcolor}
\usepackage{polski}
\usepackage[polish]{babel}
\usepackage[utf8]{inputenc}
\usepackage[T1]{fontenc}
\usepackage[utf8]{luainputenc}
\usepackage[hidelinks]{hyperref}
\usepackage[utf8]{inputenc}
\usepackage{caption}
\usepackage{indentfirst}
\usepackage{graphicx}
\usepackage{amsmath}
\usepackage{siunitx}
\usepackage{booktabs}
\usepackage{subfig}
\usepackage{pgf-pie}

\urlstyle{same}
	
\title{Wprowadzenie do eksploracji danych tekstowych w sieci WWW\\ Odpowiadanie na pytania}

\author{
Maria Konieczka, Alicja Poturała, Jakub Sikora
\affiliations
numery albumów: 283410, 283415, 283418 \\
\emails
maria.konieczka.stud@pw.edu.pl, alicja.poturala.stud@pw.edu.pl, jakub.sikora2.stud@pw.edu.pl
}

\newcommand{\todo}[1]{\textcolor{blue}{\textbf{TO DO:} #1}}
\newcommand{\opracowanyartykul}[1]{\textcolor{teal}{\textbf{Artykuł:} #1\\}}

\begin{document}
\maketitle

\section{Wprowadzenie}
\label{sec:wprowadzenie}


Zadanie odpowiadania na pytania (ang. \emph{question answering}) polega na stworzeniu oprogramowania umożliwiającego automatyczne udzielanie odpowiedzi na pytania zadawane przez człowieka w~ języku naturalnym. Zadanie to jest złożeniem problemu analizy języka naturalnego oraz wyszukiwania informacji.

Dyscyplina \emph{question answering} wyodrębniła się z~ informatyki na początku lat 60-tych XX wieku. Wtedy to zaczęły pojawiać się pierwsze systemy odpowiadające na pytania w~ języku angielskim. Początkowo systemy te były dedykowane konkretnym dziedzinom i~ często służyły jako interfejs do operacji na bazach danych. Jednym z~ pierwszych systemów QA były powstały  w~ 1961 system BASEBALL. Potrafił on odpowiadać na pytania związane z~ zeszłorocznymi rozgrywkami amerykańskiej ligi baseballowej. Dziesięć lat później w~1971 powstał system LUNAR, który potrafił odpowiadać na pytania o~ skały pobrane z~ Księżyca podczas misji Apollo11. 

Systemy otwarte, czyli odpowiadające na pytania ogólne, zaczęły być rozwijane około 20 lat później. Pod koniec 1993 roku stworzony został system START, który jako pierwszy korzystał z~ internetu jako źródła wiedzy. Od tego czasu większość znaczących systemów odpowiadania na pytania ogólne korzysta z~ zasobów sieci www. W~ 2006 roku powstała pierwsza wersja superkomputera IBM Watson, który już  w~ 2011 został zwycięzcą teleturnieju \emph{Jeopardy!} wygrywając z~ dotychczas najdłużej niepokonanym człowiekiem.

\subsection{Definicja problemu}\label{subsec:wpr:cel}
Celem projektu jest zaprojektowanie oraz zaimplementowanie własnego systemu odpowiadania na pytania. Postanowiliśmy, że stworzony przez nas system będzie odpowiadał na klika rodzajów pytań ogólnych sformułowanych w~ języku polskim. Uznaliśmy, że próba poradzenia sobie z~ analizą zdań w~ języku fleksyjnym, jakim jest język polski, będzie o~wiele ciekawsza oraz że na rynku cały czas jest niewiele takich systemów. 





\section{Definicja problemu}
\label{sec:wprowadzenie}
Celem projektu jest zaprojektowanie oraz zaimplementowanie własnego systemu odpowiadania na pytania. Postanowiliśmy, że stworzony przez nas system będzie odpowiadał na klika rodzajów pytań ogólnych sformułowanych w~języku polskim. Uznaliśmy, że próba poradzenia sobie z~analizą zdań w~języku fleksyjnym, jakim jest język polski, będzie o~wiele ciekawsza oraz że na rynku cały czas jest niewiele takich systemów. 


\section{Przegląd literatury}
\label{sec:przeglad}
\input{tex/03-przeglad-literatury.tex}

\section{Opis rozwiązania}
\label{sec:opis-rozwiazania}
Korzystając z~wiedzy i~doświadczeń wyniesionych z~sekcji~\ref{sec:przeglad} niniejszego sprawozdania, zdecydowaliśmy się na stworzenie systemu odpowiadającego na pytania ogólne, z~wyszczególnieniem pytań o~fakty dotyczące:
\begin{itemize}
	\item osób,
	\item miejsc,
	\item dat,
	\item cech wielkościowych,
	\item przedmiotów.
\end{itemize}

Bazą wiedzy systemu jest sieć WWW. Do wyszukiwania dokumentów w~sieci~WWW, wykorzystane zostały wyszukiwarki internetowe: \emph{Google\footnote{www.google.pl}}, \emph{Bing\footnote{www.bing.com}}, \emph{Yahoo\footnote{www.search.yahoo.com}} oraz \emph{DuckDuckGo\footnote{duckduckgo.com}}. Dodatkowo wprowadzona została opcja \emph{combined}, która zwraca rezultaty uzyskane ze wszystkich wymienionych wyżej stron. Zapytanie do wyszukiwarki jest tworzone za pomocą jednej z trzech strategii: \emph{singleQuery}, \emph{stopwords} lub \emph{chunks}. Pierwsza polega na bezpośrednim wyszukaniu wpisanego pytania, druga usuwa wyrazy ze Stop-listy przed rozpoczęciem wyszukiwania, ostatnia tworzy wiele zapytań poprzez podział zapytania na kawałki, tzw. \emph{chunks}. 

Podobnie jak we wcześniej omawianym systemie AskMSR~\cite{brill2002analysis}, do dalszej analizy przekazywane są jedynie \emph{snippety}, co pozwoliło na uproszczenie etapu filtracji akapitów. 

Odpowiedzią na przedstawione pytanie, jest prosta jedno-, dwu- lub trzywyrazowa odpowiedź typu \emph{Named-Entity} oraz adres URL, z której pochodzi dany \emph{snippet}. 

\subsection{Opis algorytmu}
Na rysunku~\ref{fig:algorithm-overview}, przedstawiony został ogólny schemat algorytmu odpowiadania na pytania.

\begin{figure}[h]
    \centering
    \includegraphics[width=0.7\columnwidth]{figures/WEDT-Algorytm.pdf}
    \caption{Ogólny schemat algorytmu odpowiadania na pytania}
    \label{fig:algorithm-overview}
\end{figure}

\subsubsection{Akwizycja pytania}
Pierwszym krokiem algorytmu jest akwizycja pytania od użytkownika systemu. Każde pytanie jest analizowane w~sposób indywidualny, bez przechowywania wcześniejszych pytań i~utrzymywania kontekstu. Nie zakładamy również żadnego profilowania użytkowników, ponieważ typowo proste pytania o~fakty, są ze sobą słabo powiązane. Dla każdego zapytania użytkownik wybiera strategię wyszukiwania tj. \emph{singleQuery}, \emph{stopwords} lub \emph{chunks} oraz jedną z 5 opcji wyszukiwania: \emph{Google}, \emph{Bing}, \emph{Yahoo}, \emph{DuckDuckGo} lub \emph{combined}.

\subsubsection{Klasyfikacja typu pytania}
W~kolejnym kroku następuje klasyfikacja typu pytania, a zaraz po niej określenie typu oczekiwanej odpowiedzi. Klasy pytań są definiowane przez zaimki pytające, podobnie jak w~\cite{gupta2012survey}. W naszym systemie wyróżniamy 15 rodzajów pytań. Rozpoznawane są zarówno formy podstawowe zaimków pytających, jak i formy odmienione.

Klasyfikacja typu pytania jest ściśle powiązana z pierwszym z dwóch etapów rozpoznawania typu odpowiedzi. Aby wykryć typ pytania, słowa z zapytania są kolejno sprawdzane w przygotowanym słowniku zawierającym pary typPytania-typOdpowiedzi, gdzie typPytania jest kluczem, natomiast typOdpowiedzi wartością. Słowo z zapytania, które jako pierwsze będzie miało przyporządkowaną wartość w słowniku, informuje o typie pytania. Od tej reguły istnieje jeden wyjątek, który wymusza sprawdzenie również kolejnych wyrazów z zapytania. Przy pytaniach \emph{Co ile ... ?}, typ pytania jest określony przez drugi wyraz z zapytania posiadający odpowiednik w słowniku. Rozpoznawane pary typPytania-typOdpowiedzi przedstawiono w tabeli \ref{tab:tabelaPytOdp}.

\begin{table}[h]
	\centering
	\begin{tabular}{|c|c| }
		
		 \hline
		\textbf{typPytania} & \textbf{typOdpowiedzi}  \\ \hline
		czyj? & OSOBA \\  \hline
		komu? & OSOBA \\ \hline
		kogo? & OSOBA \\ \hline
		kim? & OSOBA \\ \hline
		kiedy? & DATA \\  \hline
		ile? & WIELKOŚĆ \\  \hline
		co? & RZECZ \\  \hline
		czego? & RZECZ \\ \hline
		czym? & RZECZ \\ \hline
		skąd? & MIEJSCE \\ \hline
		dokąd? & MIEJSCE \\ \hline
		gdzie & MIEJSCE, RZECZ \\ \hline
		kto & OSOBA, MIEJSCE \\ \hline
		jak & comples \\ \hline
		który? & complex \\ \hline
		jaki? & complex  \\  \hline
	\end{tabular}
	\caption{Rozpoznawane pary typPytania-typOdpowiedzi}

\label{tab:tabelaPytOdp}

\end{table}

\subsubsection{Klasyfikacja typu oczekiwanej odpowiedzi}
Rozpoznawanie typu odpowiedzi składa się z jednego lub dwóch etapów. Etap pierwszy odbywa się równocześnie z rozpoznawaniem typu pytania, ponieważ oczekiwanym typem odpowiedzi jest wartość ze słownika, która odpowiada zaimkowi pytającemu z zapytania. Pojedyncze rozpoznawane typy odpowiedzi to: OSOBA, MIEJSCE, RZECZ, WIELKOŚĆ oraz DATA. Danemu typowi pytania może być przyporządkowana jeden lub więcej typów oczekiwanej odpowiedzi. Nie wszystkie typy pytań jednoznacznie określają oczekiwany typ odpowiedzi, dlatego niezbędne okazało się wprowadzenie drugiego etapu. Pytania \emph{Jak?}, \emph{Jaki/Jaka/Jakie?}, oraz \emph{Który/Która/Które?} są nietrywialne, ponieważ nie można jednoznacznie przyporządkować typu oczekiwanej odpowiedzi (\emph{complex}) bez dodatkowej analizy zapytania. 

Przy pytaniach \emph{Jaki/Jaka/Jakie?} oraz \emph{Który/Która/Które?} pierwszy krok do odnalezienia typu odpowiedzi polega na sprawdzeniu czy pomiędzy zaimkiem pytającym a czasownikiem, występuje rzeczownik, którego odmiana tj. liczba oraz przypadek, zgodna jest z odmianą zaimka pytającego. W tym celu wykorzystane zostało narzędzie \emph{Spejd}, które zwraca między innymi leksemy poszczególnych wyrazów jak i tagsety zawierające informacje o częściach mowy i ich odmianie. Leksem odpowiednio odmienionego rzeczownika jest następnie analizowany za pomocą narzędzia \emph{plWordNet}. Zwraca ono między innymi listę domen, do których przynależy dany rzeczownik. Domeny z \emph{plWordNetu} konwertowane są na jedną lub więcej oczekiwanych typów odpowiedzi z listy: OSOBA, MIEJSCE, RZECZ, WIELKOŚĆ, DATA lub brak dopasowania. Jeżeli do danego rzeczownika nie zostanie przyporządkowana żadna domena z listy, zapytanie nie zostanie dalej przetworzone. Jeżeli natomiast pomiędzy zaimkiem pytającym a czasownikiem nie występuje żaden pasujący rzeczownik, algorytm próbuje odnaleźć poprawnie odmieniony przymiotnik znajdujący się pomiędzy zaimkiem a czasownikiem. Gdy takowy występuje, oczekiwanym typem odpowiedzi jest WIELKOŚĆ. 
Jeżeli taki przymiotnik nie zostanie odnaleziony, ostatnią szansą jest odszukanie rzeczownika w pozostałej części zapytania tj. za czasownikiem. Odmiana rzeczownika również musi być zgodna z odmianą zaimka pytającego. Jeżeli zostanie on odnaleziony, to dalsze postępowanie jest analogiczne do sytuacji z rzeczownikiem znajdującym się pomiędzy zaimkiem a czasownikiem. Dodatkowo wyszczególniony został przypadek pytań w stylu \emph{Który z?}. Dla takiego przypadku poszukiwany jest rzeczownik w liczbie mnogiej w dopełniaczu. 

Przy pytaniu \emph{Jak?} najpierw poszukiwany jest przymiotnik znajdujący się pomiędzy zaimkiem pytającym a czasownikiem. Jeżeli taki zostanie odnaleziony, szukanym typem odpowiedzi jest WIELKOŚĆ. W przeciwnym wypadku oczekiwanym typem może być zarówno WIELKOŚĆ, OSOBA, MIEJSCE czy RZECZ. 

\subsubsection{Generator zapytań}
Generator zapytań otrzymuje pytanie wprowadzone przez użytkownika wraz ze strategią generowania zapytania. Jeżeli wybrana została strategia \emph{singlequery}, rolą generatora zapytań jest jedynie przekazanie pytania do wyszukiwarki podsumowań. W przypadku pozostałych dwóch strategii, generator modyfikuje zapytanie w celu  zwiększenia liczby uzyskanych podsumowań i zmiany zakresu poszukiwań. Pierwsza z tych strategii - \emph{stopwords} polega na usunięciu słów z zapytania, które znajdują się na stop-liście. Tym samym do wyszukiwania wysłane zostanie okrojone zapytanie, pozbawione części słów, tak aby zwiększyć zakres poszukiwań, podobnie jak w~\cite{brill2002analysis}. Innym podejściem jest zastosowanie strategii \emph{chunks}, która polega na podziale zapytania na kawałki przy pomocy narzędzia \emph{Chunker}. Pozwala ono na znalezienie granic grup nominalnych. Z odnalezionych fraz tworzone są losowe kombinacje, które następnie są przekazane jako zapytanie do wyszukiwarki. Strategia \emph{chunks} umożliwiła zwiększenie rozmiaru zbioru wyszukanych \emph{snippetów}.
 
\subsubsection{Wyszukiwanie podsumowań}
Moduł wyszukiwania podsumowań, na podstawie przekazanego zapytania lub zapytań oraz typu silnika generuje zapytanie do odpowiedniej wyszukiwarki internetowej i zwraca odnalezione \emph{snippety}.

\subsubsection{Wyszukiwanie odpowiedzi}
\todo{dopisać dużo}
Każdy wyszukany \emph{snippet} zostanie poddany analizie morfologicznej, po której przyporządkowana zostanie mu odpowiednia ocena. W~każdym zdaniu, za pomocą taggera oraz \emph{plWordNetu} wyszukiwane będą \emph{Named-Entities}, które mogą stanowić potencjalną odpowiedź na pytanie.

\subsubsection{Zwrócenie odpowiedzi}
Ostatnim krokiem procesu jest zwrócenie najlepszej odpowiedzi. Dodatkowo do odpowiedzi dołączony może zostać link ze źródłem, w~celu umożliwienia użytkownikowi systemu osobistej weryfikacji odpowiedzi.




\section{Szczegóły implementacyjne}
\label{sec:implementacja}
Przygotowany system odpowiadania na pytania został zaprojektowany w~oparciu o~modularne mikrousługi, komunikujące się ze sobą wykorzystując REST API. Takie podejście pozwoliło nam wykorzystać zewnętrzne usługi, oferujące narzędzia do przetwarzania języka naturalnego, bez ich instalacji na lokalnej maszynach. Zastosowanie architektury mikroserwisowej pozwala również na prostsze skalowanie rozwiązania oraz łatwiejsze udostępnianie systemu w~sieci WWW.

\begin{figure}[h]
    \centering
    \includegraphics[width=\columnwidth]{figures/WEDT-Uslugi.pdf}
    \caption{Ogólny schemat komunikacji pomiędzy mikrousługami w~systemie}
    \label{fig:microservices}
\end{figure}

Na rysunku~\ref{fig:microservices} przedstawiony został ogólny schemat komunikacji pomiędzy poszczególnymi usługami. Centralną częścią systemu jest moduł \texttt{KPS}. Jego rolą jest zlecanie zadań innym usługom, analiza zwróconych przez nie wyników, a przede wszystkim odpowiadanie na pytania. Został on w~całości napisany w~języku Python, wykorzystując do komunikacji bibliotekę \texttt{Flask}.

Zadawanie pytań odbywa się przy użyciu aplikacji klienckiej. Dzięki temu że moduł \texttt{KPS} wystawia REST API umożliwiające odpytywanie, aplikacją kliencką może być każdy program, który jest w~stanie obsługiwać gniazda BSD oraz wiadomości w~formacie JSON. Do normalnego użytku, moduł \texttt{KPS} udostępnia prostą aplikację kliencką możliwą do uruchomienia z~poziomu przeglądarki, natomiast do automatycznego testowania, wykorzystany został skrypt w~języku Python.

Kod realizujący zadanie ekstrakcji informacji, z~którego korzysta omawiany moduł, został zagregowany w~pakiet \texttt{algorithm}. W~skład tego pakietu wchodzą:
\begin{itemize}
    \item algorytm ekstrakcji odpowiedzi,
    \item typ domeny podstawowej,
    \item typ domeny złożonej,
    \item zestaw wyrażeń regularnych do wykrywania dat,
    \item typ n-gramowy,
    \item generator n-gramów,
    \item klasyfikator typu pytania,
    \item klasyfikator typu odpowiedzi,
    \item serwisy do parsowania odpowiedzi z~usług zewnętrznych.
\end{itemize}

Drugim kluczowym modułem, który został w~całości przygotowany przez nasz zespół, był moduł \texttt{Zapytajka}. Jego rolą w~systemie jest agregacja podsumowań z~czterech różnych wyszukiwarek internetowych. Spełnia on rolę interfejsu pomiędzy aplikacją IR a~siecią~WWW. Moduł ten, podobnie jak poprzedni, został przygotowany przy pomocy języka Python i~biblioteki \texttt{Flask}. Moduł ten udostępnia proste REST API do komunikacji \textit{proces-proces} oraz prosty interfejs graficzny do komunikacji \textit{człowiek-proces}.

Moduł \texttt{Zapytajka} do zbierania podsumowań wykorzystuje silnik Selenium. Z~powodu iż wszystkie cztery omawiane wyszukiwarki nie udostępniają prostego, a~w~szczególności darmowego API do odpytywania ich silnika, w~naszym rozwiązaniu korzystamy z~procesu \textit{webscrapingu}, czyli renderowania pobranej strony w~przeglądarce, a~następnie analizie otrzymanego dokumentu HTML, filtracji niepotrzebnych tagów i ekstrakcji szukanych fragmentów. Rozwiązanie to nie jest pozbawione wad, producenci wyszukiwarek na bieżąco zmieniają układ dokumentu HTML, tak aby uniemożliwić zbyt intensywne \textit{webscrapowanie}, dlatego też ważne jest aby stale utrzymywać kod tego modułu i na bieżąco go modyfikować.

W~momencie przyjścia pytania, na które moduł ma znaleźć potencjalne odpowiedzi, uruchamiany jest nowy proces przeglądarki Firefox. W~międzyczasie, pytanie przekazywane jest do modułu generacji zapytań. Moduł ten na podstawie przekazanego parametru, wybiera algorytm generacji nowych zapytań na podstawie oryginalnego pytania.
Zaimplementowane zostały trzy strategię:
\begin{itemize}
    \item singlequery -- brak modyfikacji oryginalnego zapytania,
    \item stopwords -- usunięcie z~zapytania tak zwanych \textit{stopwordów} w~celu zwiększenia gęstości informacji,
    \item chunks -- podział pytania na kilka powiązanych wewnętrznie fraz (chunków), z~wykorzystaniem narzędzia \textit{iobber}.
\end{itemize}
Dzięki zastosowaniu wzorca \textit{dependency injection}, rozwiązanie to jest w~bardzo prosty sposób rozszerzalne o~nowe strategię generacji zapytań. 

Przedstawione do tej pory moduły zostały w~całości przygotowane przez nasz zespół. Pozostałe usługi wykorzystywane w~systemie, zostały publicznie udostępnione i znalezione przez nas w~internecie. Najważniejszą z~nich jest usługa CLARIN, udostępniająca poszczególne narzędzia do przetwarzania języka naturalnego, również w~języku polskim. CLARIN to europejska sieć badawcza zajmująca się archiwizacją i przetwarzaniem zasobów językowych w naukach humanistycznych i społecznych. CLARIN to skrót od Common Language Resources and Technology Infrastructure.

Wykorzystując ogólnie dostępne REST API z~systemu CLARIN, nasz system wykorzystuje takie narzędzia jak:
\begin{itemize}
    \item analizator morfologiczny \textit{morfeusz},
    \item tagger \textit{morfoDita},
    \item narzędzie do NER \textit{liner2},
    \item parser składniowy \textit{spejd},
    \item chunker \textit{iobber}.
\end{itemize}

Oprócz systemu CLARIN, nasze rozwiązanie wykorzystywało również REST API udostępnione przez usługę \textit{plWordNet}, służącą do zamiany słów na odpowiadające im lemmy.

Komunikacja z~zewnętrznymi systemami została zrealizowana za pomocą specjalnie przygotowanych interfejsów dostępowych, zagregowanych w~jeden pakiet języka \texttt{Python} nazwany \texttt{services}. W~skład tego pakietu wchodzą klasy służące do komunikacji z~usługami zewnętrznymi oraz wyszukiwarkami internetowymi. Na rysunku~\ref{fig:services-classes} przedstawiony został diagram klas, który przedstawia jakie klasy zawiera ten pakiet oraz jakie relacje występują pomiędzy nimi.

\begin{figure}[h]
    \centering
    \includegraphics[width=\columnwidth]{figures/WEDT-Klasy.pdf}
    \caption{Diagram klas pakietu \texttt{services}}
    \label{fig:services-classes}
\end{figure}

\section{Instrukcja obsługi}
\label{sec:instrukcja}
Przed uruchomieniem i~przetestowaniem systemu, należy najpierw zainstalować wszystkie zależności i~uruchomić odpowiednie moduły.

\subsection{Instalacja i uruchomienie systemu}
W~celu uruchomienia obu usług, a~co za tym idzie całego systemu, należy w~pierwszej kolejności przygotować wirtualne środowisko Pythona. Wraz z~kodem źródłowym, dostarczone zostały dwa pliki \texttt{Pipfile} oraz \texttt{Pipfile.lock}. Opisują one wersję interpretera Pythona, a~także wszystkie potrzebne zależności, które muszą zostać zainstalowane przed uruchomieniem modułów. Dodatkowo, drugi plik opisuje dokładne wersje tych zależności. Dzięki zastosowaniu wirtualnego środowiska, eliminujemy problem różnych wersji zależności, już znajdujących się na maszynie docelowej. Aby utworzyć nowe środowisko, należy wykonać polecenia \texttt{pipenv~shell} oraz \texttt{pipenv~install}.

Dodatkowo, w~celu umożliwienia poprawnego działania modułu \texttt{Zapytajka}, należy zainstalować silnik przeglądarki Firefox. W~tym celu, należy wykonać polecenie \texttt{webdrivermanager firefox}.

W celu uruchomienia obu modułów, należy wykorzystać serwer wsgi \texttt{gunicorn}. Uruchomienie serwerów dzieje się przy użyciu następujących poleceń:
\begin{center}
    \texttt{gunicorn -w 4 kps:app}
\end{center}
oraz 
\begin{center}
    \texttt{gunicorn -w 4 summary\_{}search:app}
\end{center}

\subsection{Instrukcja użytkownika}
Aby skorzystać z~aplikacji klienckiej, należy uruchomić dowolną przeglądarkę oraz przejść pod adres \texttt{http://localhost:5000}. Uruchomi to stronę internetową z~prostym formularzem. Należy wybrać silnik wyszukiwania, strategię generowania zapytań i~wpisać pytanie. Po wypełnieniu formularza, należy go potwierdzić, klikając przycisk \textit{Zadaj}. Rozpocznie to proces wyszukiwania odpowiedzi. Po chwili, pojawi się kolejna strona ze znalezioną odpowiedzią oraz linkiem do dokumentu, w~którym można znaleźć więcej informacji.

\begin{figure}[h]
    \centering
    \fbox{\includegraphics[width=\columnwidth]{figures/kps.png}}
    \caption{Formularz do zadawania pytań}
    \label{fig:algorithm-overview}
\end{figure}


\section{Testy rozwiązania}
\label{sec:testy}
\subsection{Metodologia testowania}
Testowanie stworzonego systemu odbywało się w sposób półautomatyczny. Napisany został skrypt tester.py, który w sposób automatyczny wysyła zapytania do systemu z pytaniami pobranymi z pliku .xlsx. Plik wejściowy oprócz pytań zawiera także kolumnę ze spodziewanymi odpowiedziami. Po skompletowaniu odpowiedzi na wszystkie zadane pytania,  uzyskane wyniki są zapisywane do wyjściowego formularza .xlsx. Aby ułatwić proces oceny, w pliku tym oprócz odpowiedzi i pytań znajdują się także skopiowane z pliku wejściowego spodziewane odpowiedzi i parametry systemu.

Stworzenie automatycznej weryfikacji odpowiedzi jest zadaniem nietrywialnym. Ponadto trudno byłoby zagwarantować poprawność takiego narzędzia. Z tego powodu ocena poprawności udzielonych odpowiedzi odbywa się manualnie. 

\subsection{Zbiór testowy}
W~celu weryfikacji poprawności naszego rozwiązania, przygotowaliśmy zbiór danych testowych, składający się ponad 250 pytań o~ogólnej tematyce. Przygotowując pytania, staraliśmy się równomiernie podzielić zbiór pomiędzy pięć typów potencjalnej odpowiedzi. Pytania pochodziły głównie z~popularnych teleturniejów takich jak Milionerzy czy Jeden z~dziesięciu ale również z~innych popularnych quizów internetowych oraz konkursów wiedzy dla młodzieży.

\begin{figure}[h!]
    \begin{tikzpicture}
        \pie [rotate = 180]
        {20.3/OSOBA,
         19.9/MIEJSCE,
         19.6/DATA,
         20.3/WIELKOŚĆ,
         19.9/RZECZ}
    \end{tikzpicture}
    \label{fig:rozklad-typow-odpowiedzi}  
    \caption{Rozkład typów oczekiwanych odpowiedzi w~przygotowanej bazie pytań}
\end{figure}

Z~powodu charakteru pytań o~fakty, większość oczekiwanych odpowiedzi to rzeczowniki lub liczebniki. Oprócz tego, pojawią się również pytania o~cechy: przymiotniki i~przysłówki.

Duży zbiór danych pozwoli nam na przeprowadzenie badań nad poprawnością odpowiedzi systemu. Oprócz gotowego systemu, powstanie moduł badający procent poprawnych odpowiedzi w~sposób automatyczny. Aby tak badać poprawność, moduł ten musi mieć dostęp do poprawnych odpowiedzi. Co więcej, odpowiedź w~systemie powinna być przechowywana we wszystkich możliwych formach, ponieważ system może zwracać odpowiedź w~dowolnym przypadku, osobie czy liczbie. Problem ten został szerzej opisany w~artykule~\cite{brill2002analysis}.

Węzeł do automatycznego badania poprawności odpowiedzi pozwoli nam na przeprowadzenie dodatkowych badań nad zasadnością pewnych elementów w~systemie oraz wartości parametrów modułów. Modularność projektowanego systemu pozwoli nam na wyłączanie pewnych komponentów takich jak słownik synonimów, dzięki czemu będziemy mogli znaleźć optymalny kompromis pomiędzy dokładnością odpowiedzi a~czasem oczekiwania na odpowiedź.

\begin{figure}[h!]
    \begin{tikzpicture}
        \pie [rotate = 180]
        {62.5/Rzeczowniki,
         29.8/Liczebniki,
         7.7/Przymiotniki/przysłówki}
    \end{tikzpicture}
    \label{fig:rozklad-typow-odpowiedzi2}  
    \caption{Rozkład części mowy oczekiwanych odpowiedzi w~przygotowanej bazie pytań}
\end{figure}


\begin{figure}
    \begin{tikzpicture}
        \begin{axis}[
            grid=both,
            width=\columnwidth,
            title=Porównanie wyników dwóch wyszukiwarek,
            ybar,
            ymin=0,
            bar width=0.5cm,
            enlarge x limits=0.2,
            area legend,
            legend style={at={(0.5,-0.15)},
            anchor=north,
            legend columns=2},
            ylabel={Liczba odpowiedzi},
            symbolic x coords={dobra, częściowo, błędna, brak},
            xtick=data,
            nodes near coords,
            x label style={
            fixed},
            ]
        \legend{google, duckduckgo}
        \addplot coordinates {(dobra,80) (częściowo,8) (błędna,137) (brak,31)};
        \addplot coordinates {(dobra,66) (częściowo,3) (błędna,111) (brak,76)};
        
        \end{axis}
        \end{tikzpicture}
\end{figure}

\section{Wnioski i perspektywy rozwoju}
\label{sec:wnioski}






\subsection{Problematyczne aspekty systemu}

Korzystanie z zewnętrznych usług z systemu \emph{CLARIN} jak i wyszukiwarek internetowych jest jednocześnie ogromną zaletę jak i wadą naszego systemu. Wśród minusów takiego rozwiązania można wymienić to, że odpowiadanie na pytania działa wyłącznie wtedy, gdy dostępne jest połączenie internetowe oraz działają usługi \emph{CLARIN}. Dwukrotnie zdarzyło nam się, że usługi systemu \emph{CLARIN} przez cały dzień nie były dostępne, co sprawiało, że nasz system nie działał. Dodatkowo korzystanie z tych usług sprawia, że czas potrzebny na zwrócenie odpowiedzi na pytanie znacząco się wydłuża. Kolejną wadą jest też pewne uzależnienie od formy aktualnie zwracanej przez usługę. Zarówno w zewnętrznych usługach systemu \emph{CLARIN}, jak i wyszukiwarkach internetowych mogą zostać wprowadzone zmiany, które sprawią, że format zwracanych przez nie rezultatów zmieni się i nasz system może wtedy przestać poprawnie działać aż do wprowadzenia przez nas aktualizacji. 

Ryzykownym założeniem w naszym systemie jest także ograniczenie się do uni-, bi- i trigramów, co uniemożliwiło odszukanie odpowiedzi na pytanie \emph{Jak nazywali się trzej słynni bracia z legendy?}, ponieważ odpowiedź \emph{Lech, Czech i Rus} zawiera 4 wyrazy. Założyliśmy jednak, że w tak krótkich podsumowaniach prawdopodobieństwo odnalezienia powtarzających się n-gramów o długości większej niż 3 jest niskie.

Problemy występowały także przy określaniu typu oczekiwanej odpowiedzi na podstawie typu pytania. W szczególności nie zawsze wykorzystywane przez nas reguły sprawiały, że typ oczekiwanej odpowiedzi został poprawnie określony. Przykładem na to jest pytanie \emph{Pod jakim pseudonimem pisał Aleksander Głowacki?} Na podstawie rzeczownika \emph{pseudonim} nie da się określić jednoznacznie typu odpowiedzi, ponieważ pseudonimem mogłby być wyraz z domeny MIEJSCE, RZECZ jak i OSOBA.

\subsection{Perspektywy rozwoju}
Z uwagi na dużą modularność systemu, poszczególne aspekty naszego rozwiązania łatwo poddać ewentualnym przeróbkom. Po przeprowadzeniu testów zgromadziliśmy pewne przemyślenia, które poprawiłyby jakość odpowiedzi zwracanych przez nasz system. 

\subsubsection{Szukanie odpowiedzi w całych dokumentach}
Podsumowania zwracane przez wyszukiwarki nie zawsze zawierają odpowiedź na zadane pytanie. W wielu przypadkach strona, na którą wskazuje podsumowanie rzeczywiście zawiera odpowiedź na dane pytanie, jednak interesujące nas fragmenty nie znajdują się w snippecie zwróconym przez przeglądarkę. Być może dobrym dalszym krokiem w rozwoju naszego systemu byłoby zaprojektowanie kolejnego dużego modułu, który pobierałby całą zawartość dokumentu, analizował go, i zwracał tylko najbardziej istotne fragmenty.

\subsubsection{Zwracanie listy najlepiej ocenionych odpowiedzi}
Jak zauważyliśmy bardzo często poprawne odpowiedzi pojawiały się na listach z n-gramami. Powodem, dla którego nie zostały wybrane było między innymi to, że zostało zwrócone inne, popularniejsze słowo powiązane z daną tematyką. Przykładem na to jest zwrócenie wyrazu \emph{koń} zamiast \emph{cylinder} w pytaniu o rodzaj nakrycia głowy stosowany w dyscyplinie ujeżdżania lub okrągłych rocznic dat wydarzeń, zamiast rzeczywistych dat. Zwrócenie obok odpowiedzi, listy z kilkoma innymi propozycjami zwiększyłoby prawdopodobieństwo otrzymania poprawnej odpowiedzi.

\subsubsection{Przetestowanie innych podejść do rozpoznawania typu potencjalnej odpowiedzi oraz wyszukiwania odpowiedzi}
Zarówno rozpoznawanie typu potencjalnej odpowiedzi jak i wyszukiwania konkretnej odpowiedzi z podsumowań można by wykorzystać sieci neuronowe. Nie zdecydowaliśmy się na to podejście w projekcie z uwagi na to, że chcieliśmy zapoznać się z różnymi, dostępnymi rozwiązaniami dla języka polskiego, m.in. z możliwościami oferowanymi przez \emph{plWordnet} czy usługami z systemu \emph{CLARIN}, ale być może podejście z wykorzystaniem sieci neuronowych osiągałoby lepsze rezultaty.

\subsubsection{Zwiększenie liczby rozpoznawanych domen}
W dalszych krokach można by także powiększyć liczbę rozpoznawanych dziedzin poszerzając listę o kolejne pozycje, albo poprzez uszczegółowienie obecnej. Można by także zrezygnować z własnych domen i przejąć nazewnictwo domen istniejących w narzędziu \emph{plWordnet}. 

\subsubsection{Wykorzystanie innych zewnętrznych usług}
Wykorzystywane przez nas zewnętrzne usługi nie są idealne tj. nie zawierają wszystkich przypadków występujących w języku polskim. Narzędzie \emph{plWordnet} nie ma pełnej bazy wyrazów oraz nie odnajduje wszystkich znaczeń, \emph{Spejd} nie zawsze zwracał poprawny w danej sytuacji leksem i mylił się przy odmianach przez przypadki, zdarzało się także że \emph{Chunker} nieprawidłowo określał granicę wyrazów. Dlatego, aby poprawić jakość zwracanych odpowiedzi, można by zwiększyć ilość zewnętrznych usług, z których korzystamy. W szczególności rozważylibyśmy tu dodanie analizy zależności pomiędzy wyrazami, czyli budowanie drzew rozbioru. Dodatkowo można by wprowadzić pewną redundancję. Funkcjonalność niektórych zewnętrznych usług pokrywa się. Można by zwiększyć prawdopodobieństwo otrzymania prawidłowego w danej sytuacji rezultatu poprzez wprowadzenie mechanizmu głosowania. Przykładowo do wyznaczenia leksemu można by skorzystać z trzech niezależnych usług, aby na koniec przeprowadzić głosowanie w celu wybrania wspólnego, najczęściej pojawiającego się rozwiązania. 

\subsubsection{Rozszerzenie istniejących modułów}
Przy analizie wyników zaobserwowaliśmy, że w naszym systemie są elementy, które można by jeszcze udoskonalić. Jednym z takich aspektów jest uszczegółowienie i rozszerzenie listy sprawdzanych wyrażeń regularnych przy pytaniach o daty i wielkości. Przykładowo można by dodać rozpoznawanie wyrażeń takich jak jednostki, \emph{wiek} czy \emph{przed naszą erą}. Dodatkowo można by się zastanowić nad ujednoliceniem niektórych sformułowań, np. rozpoznawanie \emph{św.} oraz święty jako to samo wyrażenie. Podobny zabieg można by wprowadzić przy datach i liczbach. 

Dodatkowo można by wprowadzić większą liczbę parametrów, w szczególności rozdzielić parametr \emph{minimum\_ngram\_appearance} na osobny dla uni-, bi- i trigramów. Tak samo można by zbadać i oceniać prawdopodobieństwo z jakim dana domena występuje przy danym typie pytania. Przykładowo jak często na pytanie \emph{Gdzie?} odpowiedź będzie dotyczyła miejsca, a jak często rzeczy. Tak samo pewne wagi można by przyporządkować do wyrażeń regularnych, aby faworyzować wyszukiwanie całych dat nad zwracaniem informacji tylko o miesiącu. 

Pewne udoskonalenia można by także wprowadzić w module rozpoznawania typu oczekiwanej odpowiedzi. Obecnie w drugim etapie typ oczekiwanej odpowiedzi określany jest na podstawie analizy pojedynczego wyrazu. Być może w niektórych przypadkach nie będzie to wystarczające i niezbędne okazałoby się przeanalizowanie całej grupy wyrazów znajdujących się za zaimkiem pytającym lub chociażby znalezienie zależności pomiędzy nimi. 

Ostatnim aspektem do rozważenia jest odnalezienie narzędzia, o ile takowe istnieje, które zwróciłoby leksemy uwzględniające zwroty o długości większej niż jeden wyraz. Przykładowo obecnie nasz system ma problem z odpowiedzią na pytanie o maszynę parową, ponieważ odnajdywane są leksemy \emph{maszyna} i \emph{parowy}. Bigram \emph{maszyna parowy} nie zostanie odnaleziony w bazie \emph{WordnetPl}. Wyszukanie oryginalnej formy wyrazów również może nie zwrócić poprawnej odpowiedzi, ponieważ narzędzie \emph{WordnetPl} dla odmienionego zwrotu (przykładowo \emph{maszyną parową}) również nie zwróci żadnego wyniku.



\section{Podsumowanie}
\label{sec:podsumowanie}
\todo{krótkie pitu pitu co nam się podobało (ofc nasze rozwiazanie)}


\bibliographystyle{abbrv}
\bibliography{bibliography}

\end{document}