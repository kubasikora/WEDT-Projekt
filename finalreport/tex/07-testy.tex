\subsection{Metodologia testowania}
Testowanie stworzonego systemu odbywało się w sposób półautomatyczny. Napisany został skrypt tester.py, który w sposób automatyczny wysyła zapytania do systemu. Pytania pobierane są z arkusza kalkulacyjnego, w którym znajdują się także spodziewane odpowiedzi. Po skompletowaniu odpowiedzi na wszystkie zadane pytania,  uzyskane wyniki są zapisywane do wyjściowego formularza. Aby ułatwić proces oceny, do pliku wyjściowego oprócz zwróconej odpowiedzi zapisywane są także parametry systemu, treści pytań oraz spodziewane odpowiedzi pochodzące z pliku wejściowego.

Stworzenie automatycznej weryfikacji odpowiedzi jest zadaniem nietrywialnym. Ponadto trudno byłoby zagwarantować poprawność takiego narzędzia. Z tego powodu podjęta została decyzja o manualnym sposobie oceny poprawności odpowiedzi.

Wyróżnione zostały cztery rodzaje odpowiedzi:
\begin{enumerate}
	\item poprawne,
	\item częściowo poprawne - są to odpowiedzi, w których zabrakło detali (na przykład odpowiedź \textit{włócznia} na pytanie \textit{Co Otto III podarował Bolesławowi Chrobremu?}, gdzie poprawną odpowiedzią byłaby \textit{włócznia św. Maurycego}),
	\item brak odpowiedzi, 
	\item niepoprawne.
\end{enumerate}

Do oceny jakości systemu wykorzystano dwa wskaźniki: pokrycie i precyzję.
\begin{equation}
		coverage =  \frac{n - n_n}{n}
	\end{equation}
\begin{equation}
		precission = \frac{n_g}{n - n_n}
\end{equation}
gdzie $n$ to liczba zadanych pytań, $n_n$ to liczba pytań, na które nie została udzielona odpowiedź a $n_g$ to liczba pytań, na które została udzielona w pełni poprawna odpowiedź. 

\subsection{Zbiór testowy}
W~celu weryfikacji poprawności naszego rozwiązania, przygotowaliśmy zbiór danych testowych, składający się ponad 250 pytań o~ogólnej tematyce. Przygotowując pytania, staraliśmy się równomiernie podzielić zbiór pomiędzy pięć typów potencjalnej odpowiedzi. Pytania pochodziły głównie z~popularnych teleturniejów takich jak Milionerzy czy Jeden z~dziesięciu ale również z~innych popularnych quizów internetowych oraz konkursów wiedzy dla młodzieży.

\begin{figure}[h!]
    \begin{tikzpicture}
        \pie [rotate = 180]
        {20.3/OSOBA,
         19.9/MIEJSCE,
         19.6/DATA,
         20.3/WIELKOŚĆ,
         19.9/RZECZ}
    \end{tikzpicture}
    \label{fig:rozklad-typow-odpowiedzi}  
    \caption{Rozkład typów oczekiwanych odpowiedzi w~przygotowanej bazie pytań}
\end{figure}

Z~powodu charakteru pytań o~fakty, większość oczekiwanych odpowiedzi to rzeczowniki lub liczebniki. Oprócz tego, pojawią się również pytania o~cechy: przymiotniki i~przysłówki.

\begin{figure}[h!]
    \begin{tikzpicture}
        \pie [rotate = 180]
        {62.5/Rzeczowniki,
         29.8/Liczebniki,
         7.7/Przymiotniki/przysłówki}
    \end{tikzpicture}
    \label{fig:rozklad-typow-odpowiedzi2}  
    \caption{Rozkład części mowy oczekiwanych odpowiedzi w~przygotowanej bazie pytań}
\end{figure}

\subsection{Uzyskane wyniki}

\subsubsection{Testowanie różnych wyszukiwarek}
W ramach pierwszej serii eksperymentów sprawdziliśmy jak wybór wyszukiwarki wpływa na jakość odpowiedzi. Przeprowadzono testy dla wyszukiwarki \textit{Google} oraz \textit{DuckDuckGo}. Wybór wyszukiwarek nie był przypadkowy. Spośród dostępnych wyszukiwarek \textit{Google} zwraca najmniej podsumowań, natomiast \textit{DuckDuckGo} najwięcej. Z drugiej strony jakość streszczeń pozyskanych z \textit{Google} jest największa, a te otrzymane z \textit{DuckDuckGo} często odbiegają od tematu. Testy zostały przeprowadzone dla strategii \textit{stopwords}.

W tabelach \ref{tab:googleStopwords} i \ref{tab:DuckStopwords} oraz na wykresach \ref{fig:google-wyniki} i \ref{fig:duckduckgo-wynik} przedstawione zostały otrzymane wyniki z podziałem na typy oczekiwanych odpowiedzi. $n_g$ to liczba poprawnych odpowiedzi, $n_{pg}$ to liczba częściowo poprawnych odpowiedzi, $n_n$ odpowiada za liczbę nieudzielonych odpowiedzi, $n_w$ za liczbę niepoprawnych odpowiedzi, natomiast $cov.$ i $prec.$ to odpowiednio uzyskane pokrycie oraz precyzja.

\begin{table}[h]
	\centering
	\begin{tabular}{|c|c|c|c|c|c|c| }
	
	\hline
	\textbf{typPytania} & $n_g$ &$n_{pg}$&$n_n$&$n_w$&$cov.$&$prec.$  \\ \hline
	DATA&24&3&10&13&$\num{0.80}$&$\num{0.60}$ \\ \hline
	WIELKOŚĆ&9&1&5&37&$\num{0.90}$&$\num{0.19}$ \\ \hline
	MIEJSCE&16&0&7&28&$\num{0.86}$&$\num{0.36}$ \\ \hline
	RZECZ&12&2&3&34&$\num{0.94}$&$\num{0.25}$\\ \hline
	OSOBA&19&2&6&25&$\num{0.88}$&$\num{0.41}$\\ \hline
	\end{tabular}
	\caption{Podsumowanie wyników uzyskanych dla przeglądarki \textit{Google} oraz strategii \textit{stopwords}}
	
	\label{tab:googleStopwords}
	
\end{table}

\begin{table}[h]
	\centering
	\begin{tabular}{|c|c|c|c|c|c|c| }
		
		\hline
		\textbf{typPytania} & $n_g$ &$n_{pg}$&$n_n$&$n_w$&$cov.$&$prec.$  \\ \hline
		DATA&24&1&18&7&$\num{0.64}$&$\num{0.75}$ \\ \hline
		WIELKOŚĆ&5&1&10&36&$\num{0.81}$&$\num{0.12}$ \\ \hline
		MIEJSCE&12&0&18&21&$\num{0.65}$&$\num{0.36}$ \\ \hline
		RZECZ&11&1&14&25&$\num{0.73}$&$\num{0.30}$\\ \hline
		OSOBA&14&0&16&22&$\num{0.69}$&$\num{0.39}$\\ \hline
	\end{tabular}
	\caption{Podsumowanie wyników uzyskanych dla przeglądarki \textit{DuckDuckGo} oraz strategii \textit{stopwords}}
	
	\label{tab:DuckStopwords}
	
\end{table}

\begin{table}[h]
	\centering
	\begin{tabular}{|c|c|c|c|c|c|c| }
		
		\hline
		\textbf{przeglądarka} & $n_g$ &$n_{pg}$&$n_n$&$n_w$&$cov.$&$prec.$  \\ \hline
		\textit{Google}&80&8&31&137&$\num{0.88}$&$\num{0.36}$ \\ \hline
		\textit{DuckDuckGo}&66&3&76&111&$\num{0.70}$&$\num{0.37}$ \\ \hline
	\end{tabular}
	\caption{Porównanie otrzymanych wyników dla przeglądarek \textit{Google} oraz \textit{DuckDuckGo}}
	
	\label{tab:porownanieWysz}
	
\end{table}

W tabeli \ref{tab:porownanieWysz} znajduje się porównanie sumarycznych wyników eksperymentów dla obu przeglądarek. Warto zwrócić uwagę na spodziewaną różnicę pomiędzy pokryciem dla obu konfiguracji. Pomimo tego, że \textit{DuckDuckGo} zwraca kilkakrotnie razy więcej podsumowań, ich jakość jest stosunkowo niska i system nie odpowiedział na około $30\%$ pytań. W porównaniu dla \textit{Google} pokrycie wynosi około $88\%$. Pomimo znacznej różnicy w pokryciu, precyzje uzyskanych odpowiedzi dla obu przeglądarek są do siebie zbliżone. Wykres \ref{fig:porownanie-wyszukiwarek} zawiera porównanie sumarycznych wyników dla badanych przeglądarek.

\begin{figure}
    \begin{tikzpicture}
        \begin{axis}[
            grid=both,
            width=\columnwidth,
            ybar,
            ymin=0,
            bar width=0.5cm,
            enlarge x limits=0.2,
            area legend,
            legend style={at={(0.5,-0.15)},
            anchor=north,
            legend columns=2},
            ylabel={Liczba odpowiedzi},
            symbolic x coords={dobra, częściowo, błędna, brak},
            xtick=data,
            nodes near coords,
            x label style={
            fixed},
            ]
        \legend{google, duckduckgo}
        \addplot coordinates {(dobra,80) (częściowo,8) (błędna,137) (brak,31)};
        \addplot coordinates {(dobra,66) (częściowo,3) (błędna,111) (brak,76)};
		\end{axis}
        \end{tikzpicture}
        
        \caption{Porównanie wyników odpowiadania, bazujących na dwóch różnych wyszukiwarkach internetowych} \label{fig:porownanie-wyszukiwarek}
\end{figure}

\begin{figure}
    \begin{tikzpicture}
        \begin{axis}[
            grid=both,
            width=\columnwidth,
            ybar,
            ymin=0,
            bar width=0.2cm,
            enlarge x limits=0.12,
            area legend,
            legend style={at={(0.5,-0.15)},
            anchor=north,
            legend columns=2},
            ylabel={Liczba odpowiedzi},
            symbolic x coords={data, wielkość, miejsce, rzecz, osoba},
            xtick=data,
            nodes near coords,
            x label style={
            fixed},
            ]
        \legend{dobrze, częściowo, błędna, brak}
		\addplot coordinates {(data,24) (wielkość,5) (miejsce,12) (rzecz,11) (osoba, 14)};	
		\addplot coordinates {(data,1) (wielkość,1) (miejsce,0) (rzecz,1) (osoba, 0)};
		\addplot coordinates {(data,7) (wielkość,36) (miejsce,21) (rzecz,25) (osoba, 22)};
		\addplot coordinates {(data,18) (wielkość,10) (miejsce,18) (rzecz,14) (osoba, 16)};
		\end{axis}
        \end{tikzpicture}
        
        \caption{Szczegółowe wyniki odpowiadania przy użyciu wyszukiwarki \textit{DuckDuckGo}}\label{fig:duckduckgo-wynik}
\end{figure}

\begin{figure}
    \begin{tikzpicture}
        \begin{axis}[
            grid=both,
            width=\columnwidth,
            ybar,
            ymin=0,
            bar width=0.2cm,
            enlarge x limits=0.12,
            area legend,
            legend style={at={(0.5,-0.15)},
            anchor=north,
            legend columns=2},
            ylabel={Liczba odpowiedzi},
            symbolic x coords={data, wielkość, miejsce, rzecz, osoba},
            xtick=data,
            nodes near coords,
            x label style={
            fixed},
            ]
        \legend{dobrze, częściowo, błędna, brak}
		\addplot coordinates {(data,24) (wielkość,9) (miejsce,16) (rzecz,12) (osoba, 19)};	
		\addplot coordinates {(data,3) (wielkość,1) (miejsce,0) (rzecz,2) (osoba, 2)};
		\addplot coordinates {(data,13) (wielkość,37) (miejsce,28) (rzecz,34) (osoba, 25)};
		\addplot coordinates {(data,10) (wielkość,5) (miejsce,7) (rzecz,3) (osoba, 6)};
		\end{axis}
        \end{tikzpicture}
        
        \caption{Szczegółowe wyniki odpowiadania przy użyciu wyszukiwarki \textit{Google}}\label{fig:google-wyniki}
\end{figure}

\subsubsection{Testowanie strategii tworzenia zapytań}
Druga seria eksperymentów polegała na testowaniu systemu pod kątem wykorzystywanej strategii. Postanowiono, że każda strategia zostanie sprawdzona przy użyciu wyszukiwarki \textit{Google}. 

Uzyskane wyniki z podziałem na poszczególne typy pytań dla różnych strategii zostały przedstawione w tabelach \ref{tab:googleStopwords}, \ref{tab:GoogleSingle} oraz \ref{tab:GoogleChunks} oraz wykresach \ref{fig:google-wyniki}, \ref{fig:google-wyniki-single} i \ref{fig:google-wyniki-chunks}.

\begin{table}[h]
	\centering
	\begin{tabular}{|c|c|c|c|c|c|c| }
		
		\hline
		\textbf{typPytania} & $n_g$ &$n_{pg}$&$n_n$&$n_w$&$cov.$&$prec.$  \\ \hline
		DATA&27&3&9&11&$\num{0.82}$&$\num{0.65}$ \\ \hline
		WIELKOŚĆ&8&1&5&38&$\num{0.90}$&$\num{0.17}$ \\ \hline
		MIEJSCE&16&0&10&25&$\num{0.80}$&$\num{0.39}$ \\ \hline
		RZECZ&15&2&3&31&$\num{0.94}$&$\num{0.31}$\\ \hline
		OSOBA&19&3&7&23&$\num{0.87}$&$\num{0.42}$\\ \hline
	\end{tabular}
	\caption{Podsumowanie wyników uzyskanych dla przeglądarki \textit{Google} oraz strategii \textit{singleQuery}}
	
	\label{tab:GoogleSingle}
	
\end{table}

\begin{table}[h]
	\centering
	\begin{tabular}{|c|c|c|c|c|c|c| }
		
		\hline
		\textbf{typPytania} & $n_g$ &$n_{pg}$&$n_n$&$n_w$&$cov.$&$prec.$  \\ \hline
		DATA&18&2&19&11&$\num{0.62}$&$\num{0.58}$ \\ \hline
		WIELKOŚĆ&3&2&19&28&$\num{0.63}$&$\num{0.09}$ \\ \hline
		MIEJSCE&16&0&17&18&$\num{0.67}$&$\num{0.47}$ \\ \hline
		RZECZ&12&1&9&29&$\num{0.82}$&$\num{0.29}$\\ \hline
		OSOBA&8&1&24&19&$\num{0.54}$&$\num{0.29}$\\ \hline
	\end{tabular}
	\caption{Podsumowanie wyników uzyskanych dla przeglądarki \textit{Google} oraz strategii \textit{chunks}}
	
	\label{tab:GoogleChunks}
	
\end{table}

Porównanie uzyskanych rezultatów zostało przedstawione w tabeli \ref{tab:porownanieStrat} oraz na wykresie \ref{fig:porownanie-strategie}. Można zauważyć, że stosunkowo niewielkie pokrycie zostało otrzymane dla strategii \textit{chunks}. Otrzymany rezultat może być spowodowany wysokim zaszumieniem danych. Większość wśród rozważanych podsumowań zdecydowanie za bardzo odbiegało od tematu i dlatego system nie odpowiedział aż na 88 pytań. Najwyższa precyzja została osiągnięta dla strategii \textit{singleQuery}. Z obserwacji wynika, że wyszukiwarka \textit{Google} zwraca około ośmiu podsumowań dla każdego zapytania. Okazało się, że badanie niecałych 10. podsumowań dla pojedynczego zapytania pozwalało na uzyskanie odpowiedzi na 85 z 256 pytań. 

Porównując statystyki dla wszystkich badanych strategii, można zauważyć, że stosunkowo najgorzej system odpowiada na pytania o wielkość. Kategoria ta jest najbardziej różnorodna i stosunkowo najtrudniej jest wydobyć przedmiot zapytania.


\begin{table}[h]
	\centering
	\begin{tabular}{|c|c|c|c|c|c|c| }
		
		\hline
		\textbf{strategia} & $n_g$ &$n_{pg}$&$n_n$&$n_w$&$cov.$&$prec.$  \\ \hline
		\textit{stopwords}&80&8&31&137&$\num{0.88}$&$\num{0.36}$ \\ \hline
		\textit{singleQuery}&85&9&34&128&$\num{0.87}$&$\num{0.38}$ \\ \hline
		\textit{chunks}&57&6&88&105&$\num{0.66}$&$\num{0.34}$ \\ \hline
	\end{tabular}
	\caption{Porównanie otrzymanych wyników dla poszczególnych strategii i przeglądarki \textit{Google}}
	
	\label{tab:porownanieStrat}
	
\end{table}

\begin{figure}
	\begin{tikzpicture}
	\begin{axis}[
	grid=both,
	width=\columnwidth,
	ybar,
	ymin=0,
	bar width=0.4cm,
	enlarge x limits=0.2,
	area legend,
	legend style={at={(0.5,-0.15)},
		anchor=north,
		legend columns=2},
	ylabel={Liczba odpowiedzi},
	symbolic x coords={dobra, częściowo, błędna, brak},
	xtick=data,
	nodes near coords,
	x label style={
		fixed},
	]
	\legend{stopwords, singlequery, chunks}
	\addplot coordinates {(dobra,80) (częściowo,8) (błędna,137) (brak,31)};
	\addplot coordinates {(dobra,85) (częściowo,9) (błędna,128) (brak,34)};
	\addplot coordinates {(dobra,57) (częściowo,6) (błędna,105) (brak,88)};
	\end{axis}
	\end{tikzpicture}
	
	\caption{Porównanie wyników odpowiadania, bazujących na dwóch różnych strategiach tworzenia zapytań} \label{fig:porownanie-strategie}
\end{figure}

\begin{figure}
	\begin{tikzpicture}
	\begin{axis}[
	grid=both,
	width=\columnwidth,
	ybar,
	ymin=0,
	bar width=0.2cm,
	enlarge x limits=0.12,
	area legend,
	legend style={at={(0.5,-0.15)},
		anchor=north,
		legend columns=2},
	ylabel={Liczba odpowiedzi},
	symbolic x coords={data, wielkość, miejsce, rzecz, osoba},
	xtick=data,
	nodes near coords,
	x label style={
		fixed},
	]
	\legend{dobrze, częściowo, błędna, brak}
	\addplot coordinates {(data,27) (wielkość,8) (miejsce,16) (rzecz,15) (osoba, 19)};	
	\addplot coordinates {(data,3) (wielkość,1) (miejsce,0) (rzecz,2) (osoba, 3)};
	\addplot coordinates {(data,11) (wielkość,38) (miejsce,25) (rzecz,31) (osoba, 23)};
	\addplot coordinates {(data,9) (wielkość,5) (miejsce,10) (rzecz,3) (osoba, 7)};
	
	\end{axis}
	\end{tikzpicture}
	
	\caption{Szczegółowe wyniki odpowiadania przy użyciu strategii \textit{singleQuery} oraz wyszukiwarki \textit{Google}}\label{fig:google-wyniki-single}
\end{figure}

\begin{figure}
	\begin{tikzpicture}
	\begin{axis}[
	grid=both,
	width=\columnwidth,
	ybar,
	ymin=0,
	bar width=0.2cm,
	enlarge x limits=0.12,
	area legend,
	legend style={at={(0.5,-0.15)},
		anchor=north,
		legend columns=2},
	ylabel={Liczba odpowiedzi},
	symbolic x coords={data, wielkość, miejsce, rzecz, osoba},
	xtick=data,
	nodes near coords,
	x label style={
		fixed},
	]
	\legend{dobrze, częściowo, błędna, brak}
	\addplot coordinates {(data,18) (wielkość,3) (miejsce,16) (rzecz,12) (osoba, 8)};	
	\addplot coordinates {(data,2) (wielkość,2) (miejsce,0) (rzecz,1) (osoba, 1)};	
	\addplot coordinates {(data,11) (wielkość,28) (miejsce,18) (rzecz,29) (osoba, 19)};
	\addplot coordinates {(data,19) (wielkość,19) (miejsce,17) (rzecz,9) (osoba, 24)};
	\end{axis}
	\end{tikzpicture}
	
	\caption{Szczegółowe wyniki odpowiadania przy użyciu strategii \textit{chunks} oraz wyszukiwarki \textit{Google}}\label{fig:google-wyniki-chunks}
\end{figure}

Uzyskane rezultaty są zadowalające. Naszym założeniem było by stworzyć system, który będzie w stanie odpowiedzieć z minimum $30\%$ precyzją. Każda z badanych konfiguracji spełniła to założenie. Precyzja wahała się w przedziale $<34\%, 38\%>$. Wybór wyszukiwarki oraz strategii bardziej wpływał na pokrycie. Parametr ten wahał się od $66\%$ dla strategii \textit{chunks} do aż $88\%$ dla strategii \textit{stopwords} i wyszukiwarki \textit{Google}. 