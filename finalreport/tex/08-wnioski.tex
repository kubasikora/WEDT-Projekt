


jak typ oczekiwanejodpowiedzi określonyjest więcj niż jednym słowem to nie da się tak
zależymy od dostępności internetu i serwisów i od ich struktury

da się odszukać odpowiedź


testowanie - automatyczne testowanie 
silnik parowy - maszyna parowa
Olga Tokarczuk Wisława Szymborska
trochę inaczej godziny do napisania

do cech sięśrednio nadaje : jaki post obowiązuje w piatek ? mięsny 
fajnie bo jest modularny
lepiej sobie nie bedzie radzł bo opiera się na logice

2. jak typ oczekiwanejodpowiedzi określonyjest więcj niż jednym słowem to nie da się tak
3. maszyna parowy 
6. nie do końća sobie radzimy jak są liczby vs wyrazy i w ogóle odmiany św. święty



\subsection{Pojawiające się problemy}

daty sobie radzą najlepiej 

pytania z Jak

w ogóle język polski

niejednoznaczność odmiany

niedokładność narzędzi

który z 

co ile

to co zwraca CLARIN, wyszukiwarki tak samo

jakieś powiązanie pomiędzy focusem większym a szukanym typem odpowiedzi (domeny wordnetpl),
typ pytania nie określa typu odpowiedzi

n-gramy nie zawsze zawierają odpowiedzi, za długie są, typ odpowiedzi 

czas działania, płatne apki

\subsection{Perspektywy rozwoju}
Z uwagi na dużą modularność systemu, poszczególne aspekty naszego rozwiązania łatwo poddać ewentualnym przeróbkom. Po przeprowadzeniu testów zgromadziliśmy pewne przemyślenia, które poprawiłyby jakość odpowiedzi zwracanych przez nasz system. 

\subsubsection{Szukanie odpowiedzi w całych dokumentach}
Podsumowania zwracane przez wyszukiwarki nie zawsze zawierają odpowiedź na zadane pytanie. W wielu przypadkach strona, na którą wskazuje podsumowanie rzeczywiście zawiera odpowiedź na dane pytanie, jednak interesujące nas fragmenty nie znajdują się w snippecie zwróconym przez przeglądarkę. Być może dobrym dalszym krokiem w rozwoju naszego systemu byłoby zaprojektowanie kolejnego dużego modułu, który pobierałby całą zawartość dokumentu, analizował go, i zwracał tylko najbardziej istotne fragmenty.

\subsubsection{Zwracanie listy najlepiej ocenionych odpowiedzi}
Jak zauważyliśmy bardzo często poprawne odpowiedzi pojawiały się na listach z n-gramami. Powodem, dla którego nie zostały wybrane było między innymi to, że zostało zwrócone inne, popularniejsze słowo powiązane z daną tematyką. Przykładem na to jest zwrócenie wyrazu \emph{koń} zamiast \emph{cylinder} w pytaniu o rodzaj nakrycia głowy stosowany w dyscyplinie ujeżdżania lub okrągłych rocznic dat wydarzeń, zamiast rzeczywistych dat. Zwrócenie obok odpowiedzi, listy z kilkoma innymi propozycjami zwiększyłoby prawdopodobieństwo otrzymania poprawnej odpowiedzi.

\subsubsection{Przetestowanie innych podejść do rozpoznawania typu potencjalnej odpowiedzi oraz wyszukiwania odpowiedzi}
Zarówno rozpoznawanie typu potencjalnej odpowiedzi jak i wyszukiwania konkretnej odpowiedzi z podsumowań można by wykorzystać sieci neuronowe. Nie zdecydowaliśmy się na to podejście w projekcie z uwagi na to, że chcieliśmy zapoznać się z różnymi, dostępnymi rozwiązaniami dla języka polskiego, m.in. z możliwościami oferowanymi przez \emph{plWordnet} czy usługami z systemu \emph{CLARIN}, ale być może podejście z wykorzystaniem sieci neuronowych osiągałoby lepsze rezultaty.

\subsubsection{Zwiększenie liczby rozpoznawanych domen}
W dalszych krokach można by także powiększyć liczbę rozpoznawanych dziedzin poszerzając listę o kolejne pozycje, albo poprzez uszczegółowienie obecnej. Można by także zrezygnować z własnych domen i przejąć nazewnictwo domen istniejących w narzędziu \emph{plWordnet}. 

\subsubsection{Wykorzystanie innych zewnętrznych usług}
Wykorzystywane przez nas zewnętrzne usługi nie są idealne tj. nie zawierają wszystkich przypadków występujących w języku polskim. Narzędzie \emph{plWordnet} nie ma pełnej bazy wyrazów oraz nie odnajduje wszystkich znaczeń, \emph{Spejd} nie zawsze zwracał poprawny w danej sytuacji leksem i mylił się przy odmianach przez przypadki, zdarzało się także że \emph{Chunker} nieprawidłowo określał granicę wyrazów. Dlatego, aby poprawić jakość zwracanych odpowiedzi, można by zwiększyć ilość zewnętrznych usług, z których korzystamy. W szczególności rozważylibyśmy tu dodanie analizy zależności pomiędzy wyrazami, czyli budowanie drzew rozbioru. Dodatkowo można by wprowadzić pewną redundancję. Funkcjonalność niektórych zewnętrznych usług pokrywa się. Można by zwiększyć prawdopodobieństwo otrzymania prawidłowego w danej sytuacji rezultatu poprzez wprowadzenie mechanizmu głosowania. Przykładowo do wyznaczenia leksemu można by skorzystać z trzech niezależnych usług, aby na koniec przeprowadzić głosowanie w celu wybrania wspólnego, najczęściej pojawiającego się rozwiązania. 

\subsubsection{Rozszerzenie istniejących modułów}
Przy analizie wyników zaobserwowaliśmy, że w naszym systemie są elementy, które można by jeszcze udoskonalić. Jednym z takich aspektów jest uszczegółowienie i rozszerzenie listy sprawdzanych wyrażeń regularnych przy pytaniach o daty i wielkości. Przykładowo można by dodać rozpoznawanie wyrażeń takich jak jednostki, \emph{wiek} czy \emph{przed naszą erą}. Dodatkowo można by się zastanowić nad ujednoliceniem niektórych sformułowań, np. rozpoznawanie \emph{św.} oraz święty jako to samo wyrażenie. Podobny zabieg można by wprowadzić przy datach i liczbach. 

Dodatkowo można by wprowadzić większą liczbę parametrów, w szczególności rozdzielić parametr \emph{minimum\_ngram\_appearance} na osobny dla uni-, bi- i trigramów. Tak samo można by zbadać i oceniać prawdopodobieństwo z jakim dana domena występuje przy danym typie pytania. Przykładowo jak często na pytanie \emph{Gdzie?} odpowiedź będzie dotyczyła miejsca, a jak często rzeczy. Tak samo pewne wagi można by przyporządkować do wyrażeń regularnych, aby faworyzować wyszukiwanie całych dat nad zwracaniem informacji tylko o miesiącu. 

Ostatnim aspektem do rozważenia jest odnalezienie narzędzia, o ile takowe istnieje, które zwróciłoby leksemy uwzględniające zwroty o długości większej niż jeden wyraz. Przykładowo obecnie nasz system ma problem z odpowiedzią na pytanie o maszynę parową, ponieważ odnajdywane są leksemy \emph{maszyna} i \emph{parowy}. Bigram \emph{maszyna parowy} nie zostanie odnaleziony w bazie \emph{WordnetPl}. Wyszukanie oryginalnej formy wyrazów również może nie zwrócić poprawnej odpowiedzi, ponieważ narzędzie \emph{WordnetPl} dla odmienionego zwrotu (przykładowo \emph{maszyną parową}) również nie zwróci żadnego wyniku.



